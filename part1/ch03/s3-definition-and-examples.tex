% !TeX root = ../../book.tex
\section{定义与示例}

\subsection{``集合''的定义}

让我们从定义开始。正如我们上面开解释的那样,我们通常认为集合的特征是分组到该集合中的对象,以及使该分组有意义的属性。以下定义试图使该概念尽可能精确,同时还介绍了一些相关符号和术语。

\begin{definition}
    \textbf{集合}是具有共同明确定义属性的所有对象的整体。集合中的对象称为集合的\textbf{元素}。数学符号 ``$\in$'' 表示短语``是……的元素''(``$\notin$'' 表示``不是……的元素'')。
\end{definition}

\subsection{示例}

让我们直接用一些集合(甚至非集合)的具体例子来说明这个定义。在数学中,通常使用大写字母表示集合,使用小写字母表示集合的元素,我们通常会遵循这个约定(但并非总是如此)。为了定义或描述一个集合,我们需要识别将集合中的元素相互关联起来的共同明确定义的属性。例如,我们可以将 $B$ 定义为 NBA (美国职业篮球联赛)\footnote{原作这里用的是``美国职业棒球大联盟'',考虑到中国读者对棒球运动的陌生,译者将其改成了 NBA。}中所有球队的集合。这是一个明确定义的属性吗?如果向你展示一个对象,对于``这个对象是否具有此定义属性?''这个问题,是否有明确的``是''或``否''的答案?没错,这里就是这种情况,所以这是一个表征集合的属性。(为了避免将来读者混淆,更具体地说,$B$ 指的是 2023 赛季的 NBA 球队。)用数学语言来说,我们会写成
\[B = {\text{2023 赛季所有 NBA 球队}}\]
``大括号''—— \{ 和 \} ——表示它们之间的描述构成一个集合,其中的文本是对对象及其共同定义明确的属性的描述。现在可以说 $\text{洛杉矶湖人} \in B$ 而 $\text{多伦多哈士奇} \notin B$。

汉语中数学符号 $\in$ 的常见读法是``是……的元素''或``是……的成员''或``属于……''或``在……中''。我们主要使用``是……的元素'',因为它是其中最明确的,并且适当地使用了数学术语``\textbf{元素}''。根据上下文的不同,可以适当使用其他等效说法,但不推荐。(尤其是,``在……中''可能会与其他集合关系混淆,因此我们将完全避免使用它,并鼓励你也这样做。)

我们也已经见过一些常用的数集。过往使用这些数字的过程中你知道了它们是什么,但你可能通常不会把它们看作集合。这些集合如下:
\begin{align*}
    \mathbb{N} &= \{ \text{自然数} \} = {1, 2, 3, \dots}\\
    \mathbb{Z} &= \{ \text{整数} \} = {\dots , -2, -1, 0, 1, 2, \dots}\\
    \mathbb{Q} &= \{ \text{有理数} \}\\
      &= \{ \text{能够写成}\; \frac{a}{b} \;\text{形式的数,其中}\; a, b \in Z \;\text{且}\; b \ne 0 \}\\
      \mathbb{R} &= \{ \text{实数}\}
\end{align*}
想想上面 $\mathbb{Q}$ 的第二个定义为什么合理。很快我们就会看到一种更简洁的方式来书写类似``能够写成某某形式的数,附加额外限制条件''这种形式的句子。此外,请注意,我们未能真正定义 $\mathbb{R}$,只能说它们是实数。你要如何定义什么是实数?你尝试过吗?

\subsection{如何定义集合}

定义或描述集合的另一种方法是简单地列出其所有元素。当集合中的元质数量较少时,这种方法很方便。例如,集合 $V$ 的以下定义都是等价的:
\begin{align*}
    V &= \{A,E,I,O,U\}\\
    V &= \{\text{英语中的元音}\}\\
    V &= \{U,E,I,A,O\}
\end{align*}
``等价''的意思是,虽然上面每一行使用了不同的术语,但定义了\emph{相同的}集合 $V$。(注意:我们采用了 $y$ 是辅音的约定,因此 $y \notin V$。)与对象 $A, E, I, O, U$ 相关联的共同明确定义的属性是它们都是元音这一事实(如第二个定义所示)并且由于只有五个这样的对象,因此将它们全部列出来既简单又方便(如第一个定义所示)。

\subsubsection*{顺序和重复无关紧要}

为什么你认为第三个定义与其他定义相同?它指的是同一个对象整体,任何集合都完全由其元素来表征,因此我们编写元素的\emph{顺序无关紧要}。$U \in V$ 吗?这个问题的答案是``肯定的'',无论 $U$ 是写在元素列表的第一个还是最后一个。

不仅集合中元素的顺序无关紧要,元素的\emph{重复}也无关紧要!即,集合 $A=\{a,a,a\}$ 和集合 $A=\{a\}$ 完全相同。再次强调,集合的特征完全由它的元素决定。我们只关心集合``中''的内容。(当我们讨论集合的``口袋类比''时,我们将在第 \ref{sec:section3.4.4} 节中再次提到这一点。)写 $A = \{a, a, a\}$ 只是 $a \in A$ 重复了三次,$A$ 中只有唯一一个元素 $a$。因此,$A = \{a\}$ 是陈述相同信息的最简洁方式。

\subsubsection*{是集合的元素本身可能就是共同属性}

现在,仍然遵循可以通过编写所有元素来定义集合的思想,考虑集合 $A$ 的以下定义:
\[A=\{2, 7, 12, 888\}\]
这是当然一个集合,因为我们只是通过列出其元素来定义它。但是,关联其元素的共同明确定义的属性是什么呢?对于元音集合 $V$,我们可以列出元素并提供语言上的定义,但对于集合 $A$,我们似乎只能列出元素而不知道如何\emph{描述}其共同属性。不过,从数学上来讲,$2,7,12,888$ 的一个共同属性是它们都是集合 $A$ 的元素!在数学宇宙中,我们拥有抽象思维的自由,并且仅仅通过讨论这个集合 $A$ 及其元素,我们就赋予了它们共同的属性。这让你满意吗?你能想出\emph{另一个}常见的、明确定义的属性来精确地产生集合 $A$ 的元素吗?(提示:确定一个多项式 $p(x)$,其根恰好为 $2,7, 12, 888$。)如果集合中的元素具有多个将它们关联在一起的属性,那么你认为在引用该集合时选择哪个属性重要吗?你如何看待集合 $S := \{2, 7, \text{M}, \text{波士顿凯尔特人队}\}$?除了我们在这里列出了它们之外,是否可能存在共同属性?

\subsubsection*{省略号有时没问题,但不够正式}

有时,当所讨论的集合没有歧义,或者已经以另一种方式定义,并且我们希望列出一些元素作为说明性示例时,那么使用省略号来压缩集合元素的列表会很方便。例如,我们可以这样写
\[E = \{\text{所有偶数}\} = \{2, 4, 6, 8, 10, \dots\}\]
事实上,这个集合是个\emph{无限集},所以我们无法列出它的所有元素,但是从列出的前几个元素可以清楚地看出我们指的是偶数,更主要的是我们已经指出 $E$ 为``偶数的集合''。然而,这里必须强调,这并不是所讨论集合的精确定义。它在非正式环境中可以,但在数学上并不严格,下一小节我们讨论定义集合的正确方法时,这一点就会变得清晰起来。

\subsubsection*{集合建构符}

定义或描述集合的最佳方法是将其元素标识为具有特定属性的另一个集合的特定对象。例如,如果我们希望引用 $1$ 到 $100$(含)之间所有自然数的集合 $S$,我们可以列出所有这些元素,但这需要大量不必要的书写。我们还可以使用省略号法 $S = {1, 2, 3, \dots , 100}$,但同样,在没有 $S$ 的正式定义的情况下,这是不精确的。(有人可能会以不同的方式误解省略号。)这样写会更加精确和简洁
\[S = \{x \in \mathbb{N} \mid 1 \le x \le 100\}\]
我们将其理解为``$S$ 是自然数集 $\mathbb{N}$ 中所有 $x$ 对象的集合,满足 $1 \le x \le 100$''。

竖线符号 $\mid$ 理解为``\textbf{满足}'',表示其左边的信息告诉我们对象来自哪个``更大的集合'',右边的信息告诉我们这些对象应该具有的特定属性。

(\textcolor{red}{注意}:\emph{请勿}在其他场景下用 $\mid$ 表示满足。仅在定义集合的情况下才这么用。它只是用作占位符,将左侧(用来获取元素的集合)与右侧(这些元素应具有属性的描述)分开。)

这是非常流行且有用的\textbf{集合建构符}的示例。我们之所以这样称呼它,是因为我们正在通过从``更大''的集合中提取元素来\emph{构建}一个集合,并且只包括那些具有特定属性的元素。为此,我们需要告知读者
\begin{enumerate}[label=(\arabic*)]
    \item 更大的集合是什么;
    \item 共同属性是什么。
\end{enumerate}
让我们用几个例子来说明这个问题:
\begin{align*}
    S &= \{x \in \mathbb{N} \mid 1 \le x \le 100\} = \{1, 2, 3, \dots , 100\} \\
    T &= \{z \in \mathbb{Z} \mid \text{存在某}\; k \in \mathbb{Z} \;\text{使得}\; z = 2k\} \\
      &= \{\dots , -4, -2, 0, 2, 4, \dots\} \\
    U &= \{x \in \mathbb{R} \mid x^2 - 2 = 0\} = \{-\sqrt{2}, \sqrt{2}\}\\
    V &= \{x \in \mathbb{N} \mid x^2 - 2 = 0\}= \{ \}
\end{align*}

最后两个例子表明上下文是多么的重要。当我们更改从中提取元素的\emph{较大集合}时,相同的共同属性(满足 $x^2 -2 = 0$)可以得到不同元素的集合。两个实数满足该性质,但没有自然数满足该性质!是否有任何有理数满足该性质?你怎么认为?

这就解释了为什么指定更大的集合是绝对必要的。类似 ``$U = \{x \mid x^2 - 2 = 0\}$'' 这样的定义是\emph{没有意义的},因为它是有歧义的,可能产生完全不同的解释。

\subsubsection*{朗读建构符}

我们真的是在学习一门新的\textbf{语言},上面这些都是一些基本的词汇和语法规则。我们需要一些练习将这些句子翻译成汉语(在我们的脑海中大声读出来),反之亦然。例如,我们可以合理地将上面的 $S$ 定义为以下任何一个:

\begin{itemize}
    \item $S$ 是所有自然数 $x$ 的集合,其中 $x$ 介于 $1$ 到 $100$ 之间(含 $1$ 和 $100$)。
    \item $S$ 是 $1$ 到 $100$(含 $1$ 和 $100$)之间所有自然数的集合。
    \item $S$ 是满足不等式 $1 \le x \le 100$ 的所有自然数 $x$ 的集合。
    \item $S$ 是满足 $1 \le x \le 100$ 属性的自然数 $x$ 的集合。
\end{itemize}
请注意,它们都确定了更大的集合和共同属性;它们之间唯一的区别是语言/语法上的区别,但不会改变其数学含义。

试着为其他定义写出类似的语句。试着从朋友那里获取集合的口头定义,并用数学符号写下他们所说的内容。

考虑我们之前看过的有理数 $\mathbb{Q}$ 的定义,并注意我们可以将其重写为:
\begin{align*}
    \mathbb{Q} &= \Big\{\frac{a}{b}, \text{其中}\; a,b \in \mathbb{Z} \;\text{且}\; b \ne 0\Big\}\\
               &= \Big\{x \in \mathbb{R} \mid \text{存在}\; a, b \in \mathbb{Z} \;\text{使得}\; \frac{a}{b}= x \;\text{且}\; b \ne 0\Big\}
\end{align*}
请注意这两个定义之间的细微差别。上面一个告诉我们所有有理数都是 $\frac{a}{b}$ \textbf{的形式},然后告诉我们 $a$ 和 $b$ 必须满足的特定条件。下一个告诉我们所有有理数都是具有特定属性的实数,我们可以将该实数表示为整数之比的形式。相比之下我们更喜欢后者,因为它向我们提供了更多的信息。

一般来说,如果 $P(x)$ 表示一个描述特定明确定义的属性的句子(自然语言和/或数学语言),并且 $X$ 是给定的集合,那么符号
\[S = \{x \in X \mid P(x)\}\]
读作
\begin{center}``$S$ 是集合 $X$ 中所有元素 $x$ 的集合,使得属性 $P(x)$ 为真''。\end{center}
在符号 $P(x)$ 中,字母 $x$ 表示变量对象,根据我们输入 $x$ 的特定对象,属性 $P(x)$ 可能成立(即 $P(x)$ 为真)也可能不成立(即 $P (x)$ 为假)。如果该性质成立,则我们将 $x$ 包含在 $S$ 中(因此 $x \in S$),如果不成立,则我们不将 $x$ 包含在 $S$ 中(因此 $x \notin S$)。

回到偶数集合 $E$ 的例子,更精确的写法是
\begin{align*}
    E &= \{\text{偶数}\} \\
      &= \{x \in \mathbb{N} \mid \text{存在自然数}\; n \;\text{使得}\; x = 2n\}
\end{align*}
请注意,这里有两个属性``层''。如果我们可以找到\emph{另一个}具有 $x = 2n$ 属性的自然数 $n$,则自然数 $x$ 包含在我们的集合 $E$ 中。尝试写出奇数集或平方数集的类似定义。那么质数集呢?回文数集呢?完美数集呢?你可以使用集合构建符为这些集合编写定义吗?

\subsection{空集}

如果没有元素满足属性 $P(x)$ 怎么办?会发生什么呢?例如,考虑定义
\[S = \{x \in \mathbb{N} \mid x^2 - 2 = 0\}\]
我们知道,我们正在``寻找''的具有该属性的数字 $x$ 是 $\sqrt{2}$ (和 $-\sqrt{2}$),但是 $\sqrt{2} \notin \mathbb{N}$。因此,无论我们让 $x$ 代表 $\mathbb{N}$ 中的哪个元素,属性 $P(x)$——由 ``$x^2 - 2 = 0$'' 定义——实际上都不满足。因此,该集合中没有元素。这真的是一个集合吗?

请记住,集合完全由其元素来表征,而没有元素的集合(例如上面这个集合)则由该事实来表征。如果我们试图列出它的元素,我们最终会写成 $\{\}$。事实上,这个集合非常特别,我们给它起了一个名字和符号:

\begin{definition}
    空集是没有元素的集合。用符号 $\varnothing$ 表示。
\end{definition}

使用集合构建符定义空集的方法有很多。(是的,我们确实指的是空集;只有一个没有元素的集合!)我们在上面看到了一个例子,我们相信你可以想出许多其他例子。例如,考虑以下集合:
\begin{align*}
    &\{a \in \mathbb{N} \mid a < 0\} \\
    &\{r \in \mathbb{R} \mid r^2 < 0\} \\
    &\{q \in \mathbb{Q} \mid q^2 \notin \mathbb{Q}\}
\end{align*}
你理解为什么这些都定义了一个相同的集合,即没有元素的集合吗?

\subsubsection*{上下文相关}

我们还应再次注意到,在上面的集合构建符定义中,指定较大集合 $X$ 的重要性,我们会从该集合提取变量元素 $X$。例如,考虑以下两个集合:
\begin{align*}
    S_1 &= \{x \in \mathbb{N} \mid |x| = 5\} = \{5\}\\
    S_2 &= \{x \in \mathbb{R} \mid |x| = 5\} = \{-5, 5\}
\end{align*}
(注意:使用下标来索引集合也很常见,这允许我们反复使用相同的字母。)

在这种情况下,规范显然很重要,因为它产生了两个完全不同的集合!因此,我们定义集合时必须精确、清晰。像 $S = \{x \mid |x| = 5\}$ 这样的定义是含糊不清且不受欢迎的,因为它会导致类似上面的问题。

\subsection{罗素悖论}\label{sec:section3.3.5}

也许本节的内容看上去有点鸡蛋里挑骨头,但我们这样做背后的原因植根于集合论的一些基本思想。我们希望避免在没有这项规则的情况下可能出现的一些复杂问题和悖论。有一个十分著名的集合论悖论说明了为什么我们会有此需求,问题出在当我们使用集合构建符时,我们必须指定一个更大的集合。这个悖论被称为\emph{罗素悖论}(以英国数学家伯特兰·罗素(Bertrand Russell)命名),我们将在本节中介绍和讨论它。

\subsubsection*{集合的集合}

首先,我们需要指出,本节讨论将引入集合也可以是其他集合的元素的概念。这貌似是一个怪异且牵强的抽象想法,但它是数学中的一个基本概念。
举个具体的例子,回想一下所有 NBA 球队的集合 $B$。我们也可以将每个球队视为一个集合,其中的元素是球队中的球员。因此,可以说
\[\text{勒布朗·詹姆斯} \in \text{洛杉矶湖人队} \in B\]
因为 $\text{勒布朗·詹姆斯}$ 是集合 $\text{洛杉矶湖人队}$ 的元素,而 $\text{洛杉矶湖人队}$ 本身又是集合 $B$ 的元素。(但是请注意,$\text{勒布朗·詹姆斯} \notin B$。``$\in$'' 所表示的关系不具有\textbf{传递性}。我们将在后面定义这些术语。现在,我们说 ``$\le$'' 在实数集上表示的关系具有传递性。如果我们知道 $x \le y \le z$,那么我们可以推导出 $x \le z$。 但 ``$\in$'' 关系并非如此。)

另一个例子是 $S = \{1, 2, 3, \{10\}, \varnothing \}$。是的,空集本身可以是另一个集合的元素,集合 $\{10\}$ 也可以。为什么他们可以呢?作为思维训练,我们建议你思考一下 $\varnothing, \{\varnothing\}, \{\{\varnothing\}\}$ 之间的区别。为什么它们是不同的集合?

最后一个例子涉及自然数 $\mathbb{N}$。我们用 $\mathbb{O}$ 和 $\mathbb{E}$ 分别表示\emph{奇数}和\emph{偶数}。那么,集合 $S = \{\mathbb{O}, \mathbb{E}\}$ 是什么?它与 $\mathbb{N}$ 有何不同(如果有的话)?这是一个微妙的问题,所以要仔细思考哦。

\subsubsection*{矛盾的``集合''}

集合的集合这一概念值得花点时间仔细思考。不过,现在让我们继续讨论罗素悖论。考虑以下``集合''的定义。这里的``集合''加了引号是因为它实际上不是一个正确定义的集合,至于为什么会这样还有待考察。当我们理解它为什么不是集合后,在我们使用集合构建符时,这将成为需要指定更大集合的论据;这是因为下面的定义没有指定更大的集合。
\[\mathcal{R} = \{x \mid x \notin x\}\]
这是一个集合吗?$\mathcal{R}$ 的元素是什么?想想上面的定义所说的:$\mathcal{R}$ 的元素是恰巧不以自身为元素的集合。你能找出 $\mathcal{R}$ 的任何元素吗?你能找出不是 $\mathcal{R}$ 中元素的对象吗?

第一个问题更容易回答:到目前为止我们讨论的任何集合都是 $\mathcal{R}$ 的元素。例如,空集 $\varnothing$ 不包含任何元素,因此它本身肯定不具有元素。所以,$\varnothing \in \mathcal{R}$。另外,请注意 $\mathbb{N} \notin \mathbb{N}$(因为自然数集合本身不是自然数),所以 $\mathbb{N} \in \mathcal{R}$。

找出不是 $\mathcal{R}$ 中元素的对象是一件非常棘手的事情,我们通过提出以下问题来帮助你思考:$\mathcal{R}$ 本身是一个元素吗? $\mathcal{R} \in \mathcal{R}$ 是真是假?在继续阅读之前请先仔细思考这一点。我们将引领你如何正确的思考。

\begin{itemize}
    \item 假设 $\mathcal{R} \in \mathcal{R}$ 为真 \\
    $\mathcal{R}$ 的定义属性告诉我们,它的任何元素都是一个不以自身为元素的集合。由此,我们可以推导出 $\mathcal{R} \notin \mathcal{R}$。\\
    等一下!知道 $\mathcal{R} \in \mathcal{R}$ 使我们推导出 $\mathcal{R} \notin \mathcal{R}$。当然,这两个矛盾的事实不能同时成立。因此,一定是我们原来的假设有问题,所以一定是 $\mathcal{R} \notin \mathcal{R}$。
    \item 假设 $\mathcal{R} \notin \mathcal{R}$ 为真 \\
    $\mathcal{R}$ 的定义属性告诉我们,任何不是 $\mathcal{R}$ 元素的对象都必须是其自身的元素。(否则,它会被包含为 $\mathcal{R}$ 的元素。)因此,我们可以推导出 $\mathcal{R} \in \mathcal{R}$。\\
    等一下!知道 $\mathcal{R} \notin \mathcal{R}$ 使我们推导出 $\mathcal{R} \in \mathcal{R}$。这也是矛盾的。
\end{itemize}
无论我们选择哪一个—— $\mathcal{R} \in \mathcal{R}$ 还是 $\mathcal{R} \notin \mathcal{R}$——我们都会发现另一个也一定为真,然而这些相互矛盾的事实不可能同时为真。

这就是\textbf{悖论}。$\mathcal{R}$ 不是一个正确定义的集合。如果 $\mathcal{R}$ 是集合,我们就会发现自己陷入刚刚看到的两难境地,而这两种情况都不为真。而 $\mathcal{R}$ 也不只是空集 $\varnothing$;所以唯一的可能是 $\mathcal{R}$ 不是集合。

\subsubsection*{``所有集合的集合''\emph{不是}集合}

我们能否以某种方式修改 $\mathcal{R}$ 的定义,以产生该定义试图描述的``集合''?我们应该从哪个``更大的集合''中提取对象 $x$ ,以确保定义有意义并正确定义集合?

回顾一下我们对 $\mathcal{R}$ 定义的中文解释:``$\mathcal{R}$ 的元素是恰巧不以自身作为元素的集合。''我们需要检验所需属性 ($x \notin x$) 的对象 $x$ 实际上都是集合。那么,也许我们应该将 $X$ 定义为所有集合的集合,并使用短语 ``$x \in X$'' 作为 $\mathcal{R}$ 定义的一部分。这样不就解决了吗?
\[\mathcal{R} = \{x \in X \mid x \notin x\}\]

不,完全不是这样!\textbf{``所有集合的集合''本身并不是一个集合}。如果是的话,这将导致我们陷入与之前完全相同的悖论!唯一的区别在于我们会明确指出``更大的集合'',从中我们可以得到之前隐式指定的对象 $x$。

主要问题是,不指定从中提取对象的``更大的集合'',或者隐式引用``所有集合的集合'',会导致这种令人讨厌的悖论。因此,我们决不能允许这样的定义。任何试图从``所有集合的集合''中提取对象 $x$ 来定义一个集合,无论是隐式的还是显式的,都不是集合的正确定义。

\subsubsection*{进一步探讨}

不过,``$x \notin x$'' 给出的属性 $P(x)$ 并没有本质上的错误。问题在于我们使用的``更大的集合''。例如,拿下面这个集合来说,
\[S = \Bigg\{x \in \bigg\{\frac{1}{2}, \frac{3}{4}, \frac{5}{2}\bigg\} \mid x \notin x \Bigg\}\]
它的元素是什么?唯一的可能性是从更大的集合 $\{\frac{1}{2}, \frac{3}{4}, \frac{5}{2}\}$ 中提取的元素。请注意,这些数字都不是包含自身作为元素的集合。因此,这是集合 $\{\frac{1}{2}, \frac{3}{4}, \frac{5}{2}\}$ 本身的正确定义!根据前面 $\mathcal{R}$ 的定义,我们试图定义的对象在其自己的定义中被允许作为变量对象 $x$ 之一,这就是问题产生的地方。

可能我们稍稍偏离了最初讨论的主题,但我们认为重要的是要指出,有可能构建不明确定义的``集合'',而这些集合不是数学意义上的集合。在大多数情况下,我们在本书中使用的集合不会遇到此类问题,但掩盖这些问题或根本置之不理对作为学生的你来说不公平。如果你发现自己对这些问题感兴趣,可以找一本关于集合论的入门书来阅读。

``集合''的定义也有其他形式的错误,但接下来的例子来源于语言问题,而非数学基础出了问题,如罗素悖论。例如,我们可以说``设 $N$ 为 20 世纪所有经典小说的集合''。``经典小说''并不是一个明确定义的属性,无法用来确定集合中的元素。``经典''的概念是主观的,并不是严格精确的。此外,我们还可以说``设 $B$ 为明天出生的人的集合'',但定义中的这种时间依赖性使我们永远无法真正知道 $B$ 的元素是什么。当明天到来时,明天指的将是第二天,依此类推。你能举出其他形式不正确的元素``集合''的例子吗?你能想出像上面那样的悖论吗?

总的来说,以下陈述是从罗素悖论的讨论中得出的最重要的思想:

\begin{center}根据约定的集合规则(集合论公理),\textbf{不存在}所有集合的集合。\end{center}

\subsection{标准集及其符号}

我们已经引用并使用了一些常见的数集,现在我们列出这些数集及其标准符号:

\begin{center}
\begin{itemize}
    \item \emph{自然数}:$\mathbb{N} = \{1, 2, 3, 4, \dots \}$
    \item \emph{前 $n$ 个自然数}:$[n] := \{1, 2, 3, \dots , n-1, n \}$
    \item \emph{整数}:$\mathbb{Z} = \{\dots, -3, -2, -1, 0, 1, 2, 3, \dots \}$
    \item \emph{有理数}:$\mathbb{Q} = \{\frac{m}{n} \mid m,n \in \mathbb{Z} \;\text{且}\; b \ne 0 \}$
    \item \emph{实数}:$\mathbb{R}$
    \item \emph{复数}:$\mathbb{C}$
\end{itemize}
\end{center}

我们已经使用过 $\mathbb{N}$ 和 $\mathbb{Z}$ 好几次了。有理数 $\mathbb{Q}$(我们使用 $\mathbb{Q}$ 是因为 $\mathbb{R}$ 已被占用,并且有理数就是商(Quotient),因此取其首字母 $\mathbb{Q}$)是所有分数或整数比,包括正数和负数。实数就更难描述了。为什么我们不能像列出 $\mathbb{N}$ 和 $\mathbb{Z}$ 那样列出其元素?为什么 $\mathbb{R} \ne \mathbb{Q}$?目前,我们基本上认为我们对这些数集的知识是理所当然的,但还是请思考一下。(我们提到复数 $\mathbb{C}$ 是因为你可能熟悉它们,但我们不会在本书中使用复数。)

我们怎么知道像 $\mathbb{N}$ 这样的集合存在?为什么我们将 $\mathbb{R}$ 视为数轴?与 $\mathbb{N}$ 相比,$\mathbb{Z}$ ``多出''多少元素?与 $\mathbb{Q}$ 相比,$\mathbb{R}$ ``多出''多少元素?我们能回答这些问题吗?在不久的将来,我们将严格推导集合 $\mathbb{N}$ 并证明它是唯一具有特定属性的集合。当我们回到数学归纳法的研究时,这一点至关重要。(还记得那一章我们的目标吗?)

\subsection{问题与练习}

\subsubsection*{提醒自己}

口头或书面简要回答以下问题。这些题目全都基于你刚刚阅读的部分,因此如果你无法想起特定的定义、概念或示例,请返回重新阅读相应部分。确保自己在继续之前可以自信地回答这些问题,这将有助于你的理解和记忆!

\begin{enumerate}[label=(\arabic*)]
    \item 符号 ``$\in$'' 表示什么意思?
    \item 如何读出 ``$x \in S$'' 这句话?
    \item 一个集合有可能成为另一个集合的元素吗?如果能,请举个例子。\\
    一个集合有可能是其自身的一个元素吗?
    \item 如何读出 ``$\{x \in N \mid x \le 5\}$''?你能列出这个集合的元素吗?
    \item 这个集合是什么:$\{z \in \mathbb{Z} \mid z \in \mathbb{N}\}$?
    \item 这个集合是什么:$\{x \in [10] \mid x \ge 7\}$?
    \item 对于以下每个集合,请说明它们有多少个元素:
        \begin{enumerate}[label=(\alph*)]
            \item $\varnothing$
            \item $\{1, 2, 10\}$
            \item $\{1, \varnothing\}$
            \item $\{\varnothing\}$
        \end{enumerate}
    \item $x \in \{ 1, 2, \{x\} \}$ 吗?$\{x\} \in \{ 1, 2, \{x\} \}$ 吗?
    \item 设 $A = \{a, b, c\}, B = \{b, c, a\}, C = \{a, a, b, c, a, b\}$。这些集合是否相等?
    \item $\mathbb{Z} = \mathbb{Q}$ 吗?为什么相等或者为什么不等?
\end{enumerate}

\subsubsection*{试一试}

尝试回答以下简答题。这些题目要求你实际动笔写一写,或(对朋友/同学)口头描述一些东西。目的是让你练习使用新概念、定义和符号。别担心,这些题本来就很简单。确保能够解决这些问题将对你有所帮助!

\begin{enumerate}[label=(\arabic*)]
    \item 使用集合构建符写出 $4$ 的倍数自然数集合的定义。
    \item 考虑集合 $S = \{3, 4, 5, 6\}$。使用集合构建符以两种不同的方式定义 $S$。
    \item 给出一个满足 $\mathbb{N} \in \mathbb{X}$ 但 $\mathbb{Z} \notin \mathbb{X}$ 的集合 $X$ 的例子。
    \item 给出一个包含 $100$ 个元素的集合的示例。
    \item 给出一个集合 $A, B, C$ 的例子,使得 $A \in B$ 且 $B \in C$ 但 $A \notin C$。
    \item 使用集合构建符编写奇数集合的定义。
    \item 使用集合构建符编写非自然数整数集合的定义。
    \item 考虑以下集合:
        \begin{align*}
            A &= \{x \in \mathbb{R} \mid x^2 - 3x + 2 \ge 0\} \\
            B &= \{y \in \mathbb{R} \mid y \le 1 \;\text{或}\; y \ge 2\}
        \end{align*}
        解释为什么 $A=B$。
    \item 考虑以下集合:
        \begin{align*}
            C &= \{x \in \mathbb{R} \mid x^2 - 4 \ge 0\} \\
            D &= \{y \in \mathbb{R} \mid y \ge 2\}
        \end{align*}
        $C = D$ 吗?为什么相等或为什么不等?使用 $\in$ 和 $\notin$ 用良好的数学符号写下你的解释。
    \item 尝试向朋友解释罗素悖论,即使是一个没有学习数学的朋友。他们对此有何理解?他们会反驳你吗?他们的想法合理吗?和朋友讨论一下!
\end{enumerate}