% !TeX root = ../../book.tex
\section{``集合''思想}

\subsubsection*{``物以类聚''}

集合的直观概念对你来说可能并不陌生。如果你奥特曼卡收藏者\footnote{原作这里用的是``棒球明星卡'',考虑到中国读者对棒球运动的陌生,译者将其改成风靡中国(青少年界)的奥特曼卡。},拥有``全套''卡片意味着拥有发行商发行的某个系列的每一张卡。如果你和朋友一起玩桌游,你们会在玩之前商定一套``规则'',这样以后就不会出现未解决的争议。如果你在生物、化学或物理课上进行了实验室实验,你会将数据收集到``数据集''中并分析这些结果以检验假设。

这是三种不同情况,每种情况都涉及单词``集或套''(\textbf{set}),那么该单词是如何关联上下文并赋予正确含义的呢?本质上,集合是指基于某些共同属性而组织在一起的对象的全体。在第一个示例中,稀有度为 UR 的每一张卡都属于该特定集合。在第二个示例中,任何商定好的规则都将属于规则集合。在第三个示例中,实验中收集的任何数据都属于该数据集。在每种情况下,都有一个共同的属性,让我们可以将特定对象彼此关联起来,并将它们作为一个集合来引用。

\subsubsection*{数学中的集合}

集合在数学中非常常见、非常流行、同时也非常有用、非常基础。因为数学家研究的是抽象对象以及这些对象之间的关系,因此如果无法引用一组数学对象,就很难准确描述所思考的内容。事实上,我们已经不自觉地用到了集合!

例如,在研究多项式和二次函数求根公式时,我们提到具有负判别式(当 $\frac{b^2}{4a} - c < 0$ 时)的二次多项式 $p(x) = ax^2 + bx + c$ \textit{在实数集中}没有根。我们想表达什么?你理解这句话吗?我们试图传达这样的想法:无论我们从所有实数的集合中选择哪个实数 $x$,都可以保证 $p(x) \ne 0$。但是实数地集合到底是什么?它是如何定义的?我们怎么能确定它存在呢?实际上这是相当难回答的问题,尝试解答这些问题会让我们远离集合论的世界。

在数学的语言中,我们的目标是使我们的句子和陈述\textit{准确无误},并寻求基于某些基本假设来建立真理。我们需要以这些假设为起点,否则我们就没有任何真理为基础。这些假设,就像每个人在``玩数学游戏''之前都同意其成为``规则集合''的一部分,被称为\textbf{公理}。

如果你学过一些几何或者读过希腊数学家欧几里得(Euclid)和他的名著《\textit{几何原本}》,那么你可能对``公理''一词不陌生。欧几里得\textit{证明}的所有基本几何结论都建立在几个基本假设之上:任意两点都可以用线段连接,必须存在给定中心点和半径的圆,非平行线相交,等等。这些陈述一开始就被认为是真的。

\textbf{集合论}作为一个重要的数学分支也构建在公理之上。集合论的公理体系为所有涉及集合的结论打下了坚实的基础,利用这些公理和由公理推导出来的结果,我们可以继续发现数学宇宙中新的真理。不过,研究这些公理及其推论更适合专门讨论集合论的课程,我们这里把集合论公理的许多推论视为理所当然,而无需严格证明它们。这并不是因为不能证明,而仅仅是因为这些证明需要占用本书太多时间和篇幅来完成。

我们\textit{要}做的是提供一个``集合''的定义,满足我们在本书中使用集合的上下文需求。我们还将定义集合的一些基本属性,分享一些说明性示例,并讨论集合上创建新集合的不同操作。

