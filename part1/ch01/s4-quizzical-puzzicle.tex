% !TeX root = ../../book.tex
\section{封面}

\subsection{封面个性化}

从 3.10 版本开始,封面更加弹性化,用户可以自行选择输出的内容,包括 \lstinline{\title} 在内的所有封面元素都可为空。目前封面的元素有

\begin{table}[htbp]
  \centering
  \caption{封面元素信息}
  \begin{tabular}{p{0.07\textwidth}p{0.15\textwidth}|p{0.07\textwidth}p{0.15\textwidth}|p{0.07\textwidth}p{0.15\textwidth}}
    \toprule
    信息 & 命令 & 信息 & 命令 & 信息 & 命令 \\
    \midrule
    标题 & \lstinline|\title| & 副标题 & \lstinline|\subtitle| & 作者 & \lstinline|\author| \\
    机构 & \lstinline|\institute| & 日期 &  \lstinline|\date| & 版本 & \lstinline|\version| \\
    箴言 & \lstinline|\extrainfo| & 封面图 & \lstinline|\cover| & 徽标 & \lstinline|\logo| \\
    \bottomrule
  \end{tabular}
\end{table}

另外,额外增加一个 \lstinline{\bioinfo} 命令,有两个选项,分别是信息标题以及信息内容。比如需要显示{\kaishu User Name:111520},则可以使用 
\begin{lstlisting}
\bioinfo{User Name}{115520}
\end{lstlisting}

封面中间位置的色块的颜色可以使用下面命令进行修改:
\begin{lstlisting}
\definecolor{customcolor}{RGB}{32,178,170}
\colorlet{coverlinecolor}{customcolor}
\end{lstlisting}

\subsection{封面图}

本模板使用的封面图片来源于 \href{https://pixabay.com/en/tea-time-poetry-coffee-reading-3240766/}{pixabay.com}\footnote{感谢 China\TeX{} 提供免费图源网站,另外还推荐 \href{https://www.pexels.com/}{pexels.com}。},图片完全免费,可用于任何场景。封面图片的尺寸为 $1280 \times 1024$, 更换图片的时候请\textbf{严格}按照封面图片尺寸进行裁剪。推荐一个免费的在线图片裁剪网站 \href{https://www.fotor.com/cn}{fotor.com}。用户 QQ 群内有一些合适尺寸的封面,欢迎取用。

\subsection{徽标}

本文用到的 Logo 比例为 1:1,也即正方形图片,在更换图片的时候请选择合适的图片进行替换。

\subsection{自定义封面}

另外,如果使用自定义的封面,比如 Adobe illustrator 或者其他软件制作的 A4 PDF 文档,请把 \lstinline{\maketitle} 注释掉,然后借助 \lstinline{pdfpages} 宏包将自制封面插入即可。如果使用 \lstinline{titlepage} 环境,也是类似。如果需要 2.x 版本的封面,请参考 \href{https://github.com/EthanDeng/etitlepage}{etitlepage}。

\section{章标标题}

本模板内置 2 套\textit{章标题显示风格},包含 \lstinline{hang}(默认)与 \lstinline{display} 两种风格,区别在于章标题单行显示(\lstinline{hang})与双行显示(\lstinline{display}),本说明使用了 \lstinline{hang}。调用方式为
\begin{lstlisting}
\documentclass[hang]{elegantbook} %or
\documentclass[titlestyle=hang]{elegantbook}
\end{lstlisting}

在章标题内,章节编号默认是以数字显示,也即{\kaishu 第 1 章},{\kaishu 第 2 章}等等,如果想要把数字改为中文,可以使用
\begin{lstlisting}
\documentclass[chinese]{elegantbook} %or
\documentclass[scheme=chinese]{elegantbook}
\end{lstlisting}