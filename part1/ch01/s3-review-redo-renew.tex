% !TeX root = ../../book.tex
\section{颜色主题}

本模板内置 5 组颜色主题,分别为 \textcolor{structure1}{\lstinline{green}}\footnote{为原先默认主题。}、\textcolor{structure2}{\lstinline{cyan}}、\textcolor{structure3}{\lstinline{blue}}(默认)、\textcolor{structure4}{\lstinline{gray}}、\textcolor{structure5}{\lstinline{black}}。另外还有一个自定义的选项  \lstinline{nocolor}。调用颜色主题 \lstinline{green} 的方法为 
\begin{lstlisting}
\documentclass[green]{elegantbook} %or
\documentclass[color=green]{elegantbook}
\end{lstlisting}


\begin{table}[htbp]
  \caption{ElegantBook 模板中的颜色主题\label{tab:color thm}}
  \centering
  \begin{tabular}{ccccccc}
  \toprule
    & \textcolor{structure1}{green} 
    & \textcolor{structure2}{cyan} 
    & \textcolor{structure3}{blue}
    & \textcolor{structure4}{gray} 
    & \textcolor{structure5}{black} 
    & 主要使用的环境\\
  \midrule
    structure & \ccr{structure1}
    & \ccr{structure2}
    & \ccr{structure3} 
    & \ccr{structure4} 
    & \ccr{structure5} 
    & chapter \ section \ subsection \\
    main      & \ccr{main1}
    & \ccr{main2}
    & \ccr{main3}
    & \ccr{main4}
    & \ccr{main5}
    & definition \ exercise \ problem \\
    second    & \ccr{second1}
    & \ccr{second2}
    & \ccr{second3}
    & \ccr{second4}
    & \ccr{second5}
    & theorem \ lemma \ corollary\\
    third     & \ccr{third1}
    & \ccr{third2}
    & \ccr{third3}
    & \ccr{third4}
    & \ccr{third5}
    & proposition\\
  \bottomrule
  \end{tabular}
\end{table}

如果需要自定义颜色的话请选择 \lstinline{nocolor} 选项或者使用 \lstinline{color=none},然后在导言区定义 structurecolor、main、second、third 颜色,具体方法如下:
\begin{lstlisting}[tabsize=4]
\definecolor{structurecolor}{RGB}{0,0,0}
\definecolor{main}{RGB}{70,70,70}    
\definecolor{second}{RGB}{115,45,2}    
\definecolor{third}{RGB}{0,80,80}
\end{lstlisting}