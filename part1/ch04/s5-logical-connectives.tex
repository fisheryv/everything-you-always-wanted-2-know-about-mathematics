% !TeX root = ../../book.tex
\section{逻辑连词}

为了从简单陈述(即仅由量词和命题组成的数学数学)构建复杂数学陈述,我们可以将多个陈述用某些单词和短语(例如``与''、``或''和``蕴涵'')连接起来,以创建更复杂的陈述并断言进一步的主张和真理。我们将这些单词和短语称为\textbf{逻辑连词},每个词和短语都有自己相应的数学符号和含义。基于我们对语言和理性思维的直觉掌握,这些含义对你来说是直观合理的,但我们强调,引入数理逻辑及其相应符号的主要目标之一是将这些直觉构建为严格且明确的概念。

在本节中,我们假设 $P$ 和 $Q$ 是任意数学陈述。这些陈述本身可以由量词和其他连词以及各种数学概念组合组成。关键是我们将 $P$ 和 $Q$ 组成更大陈述的方式独立于它们各自的组成。之前,我们看到 ``$\neg(\forall x \in X \centerdot R(x))$'' 等价于 ``$\exists x \in X \centerdot \neg R(x)$'',无论陈述 $R(x)$ 是什么以及它有多复杂。这里延续了这一思想。我们可以讨论如何组合两个陈述,而无需单独了解它们是什么。

我们还应该指出,这些成分陈述 $P$ 和 $Q$ 实际上可能是变量命题。例如,我们将研究如何连接两个变量命题 $P(x)$ 和 $Q(x)$,每个命题都依赖于某个变量 $x$。我们在本节中发展的定义和方法适用于这些变量命题,即使这些命题本身在不知道变量 $x$ 的值是什么的情况下没有真值。

当我们想要有意义地、数学地讨论这些命题时,我们必须\textbf{量化}变量 $x$。因此,如果我们有变量命题 $P(x)$ 和 $Q(x)$,我们仍然可以有意义地定义 $P(x) \land Q(x)$(其中 $\land$ 表示``逻辑与'',正如你将在下一节中看到的)。然后,我们可以在一个例子或一个问题中讨论以下形式的声明
\[\exists x \in X \centerdot P(x) \land Q(x)\]
这是一个数学\textbf{陈述}。

本质上,我们想要表达的观点是,这些连词仍然适用于变量命题,但必须在整体陈述中的\emph{某处}对相关变量进行量化,以使变量命题成为正确的\textbf{数学陈述}。

\subsection{与}

说
\begin{center}
    ``$P$ 和 $Q$'' 为\verb|真|
\end{center} 
意味着这两个陈述都具有真值:\verb|真|。如果陈述 $P$ 或 $Q$ 之一为\verb|假|,则陈述 ``$P$ 和 $Q$'' 也为假。下面的定义概括了这个想法:

\begin{definition}
    我们在两个数学陈述之间使用符号 ``$\land$'' 来表示``和''。 例如,我们将 ``$P \land Q$'' 读作 ``$P$ 和 $Q$''。

    这称为 $P$ 和 $Q$ 的合取。

    当 $P$ 和 $Q$ 都为真时,``$P \land Q$'' 的真值为\verb|真|,否则真值为\verb|假|。
\end{definition}

以下是此定义的一些示例:

\begin{example}
    \begin{align*}
        (1 + 3 = 4) \land (\forall x \in \mathbb{R} \centerdot x^2 \ge 0) \qquad &\text{真} \\
        (1 + 3 = 5) \land (\forall x \in \mathbb{R} \centerdot x^2 \ge 0) \qquad &\text{假} \\
        (1 + 3 = 5) \land (\exists x \in \mathbb{Q} \centerdot x^2 = 2)   \qquad &\text{假}
    \end{align*}
\end{example}

\subsubsection*{符号:括号}

有时删除我们在上面示例中使用的括号是很常见的。例如,上例中的第一行可以等效地写为
\[1 + 3 = 4 \land \forall x \in \mathbb{R} \centerdot x^2 \ge 0\]
使用括号往往会使陈述更具可读性。如果没有括号,我们必须花一些额外的时间思考语句的一部分在哪里结束以及下一部分从哪里开始,但我们最终仍然可以理解它。当括号使陈述更容易理解时,我们都应当使用括号。

\subsubsection*{符号:集合和逻辑}

你可能会注意到逻辑连词 ``$\land$'' 和集合运算符 ``$\cap$'' 之间的相似性。这不是巧合!

正如我们将在 \ref{sec:section4.5.4} 节中讨论的那样,根据 ``$\cap$'' 集合运算符的底层逻辑,我们可以使用连词 ``$\land$'' 来编写 ``$\cap$'' 的定义。现在就尝试一下,然后如果你愿意的话,可以简单浏览一下该部分内容。但一般来说,请小心区分这两个符号!如果 $A$ 和 $B$ 为集合,则 ``$A \land B$'' 没有明确定义;正确含义应该是 ``$A \cap B$''。

\subsection{或}

说
\begin{center}
    ``$P$ 或 $Q$'' 为\verb|真|
\end{center}

表示 ``$P$ 为\verb|真|,或 $Q$ 为\verb|真|''。我们需要知道其中一个陈述为\verb|真|才能声明整个陈述的真值为\verb|真|。我们不关心 $P$ 和 $Q$ 是否\emph{都}为\verb|真|,只关心其中\emph{至少一个}为\verb|真|。

这与计算机科学中所谓的 ``异或''(也称为 \verb|XOR|)不同,当 $P$ 和 $Q$ 都为\verb|真| 时,``$P$ \verb|XOR| $Q$'' 为\verb|假|。在数学中,我们使用\textbf{同``或''}。我们只关心其中至少一项陈述是否成立。

\begin{definition}
    我们在两个数学陈述之间使用符号 ``$\lor$'' 来表示 ``或''。例如,我们将 ``$P \lor Q$'' 读作 ``$P$ 或 $Q$''。

    这称为 $P$ 和 $Q$ 的析取。

    当 $P$ 和 $Q$ 中至少一个为\verb|真|时(即使两者都为\verb|真|),``$P \lor Q$'' 的真值为\verb|真|,否则真值为\verb|假|。
\end{definition}

\begin{example}
    \begin{align*}
        (1 + 3 = 4) \lor (\forall x \in \mathbb{R} \centerdot x^2 \ge 0) \qquad &\text{真} \\
        (1 + 3 = 5) \lor (\forall x \in \mathbb{R} \centerdot x^2 \ge 0) \qquad &\text{真} \\
        (1 + 3 = 5) \lor (\exists x \in \mathbb{R} \centerdot x^2 < 0)   \qquad &\text{假}
    \end{align*}
\end{example}

\subsubsection*{符号}

我们在上一小节中提到的有关符号的注释同样也适用于此。首先,使用括号(如上面的示例所示)很有帮助,但在技术上不是必需的。不过,只要有帮助,我们就应当使用它们。

其次,你可能会注意到逻辑连词 ``$\lor$'' 和集合运算符 ``$\cup$'' 之间的相似性。再次强调,这不是巧合!尝试使用 ``$\lor$'' 重写 ``$\cup$'' 的定义,并简要浏览一下第 \ref{sec:section4.5.4} 节。但一般来说,请小心区分这两个符号!如果 $A$ 和 $B$ 为集合,则 ``$A \lor B$'' 没有明确定义;正确含义应该是 ``$A \cup B$''。

\subsection{条件陈述}

这是最难使用的逻辑连词,始终问题不断,因此我们要对此格外小心和明确。当 $Q$ 的真值\emph{必然}从 $P$ 的真值推导出来时,我们希望陈述``\textbf{如果} $P$,\textbf{则} $Q$''(有时写为``$P$ 蕴含 $Q$'')的真值为\verb|真|。也就是说,我们希望该陈述为\verb|真|如果以下成立:

\begin{center}
    每当 $P$ 为\verb|真|时,$Q$ 也为\verb|真|。
\end{center}

\subsubsection*{真值表和定义}

由于这是语义上最难弄清楚的连词,我们引入\textbf{真值表}的概念,让概念更容易理解:
\begin{center}
    \begin{tabular}{c|c|c|c|c|c|c}
          $P$      & $Q$      & $\neg P$ & $P \land Q$ &  $P \lor Q$ & $P \implies Q$ & $Q \implies P$\\
          \hline
          \verb|T| & \verb|T| & \verb|F| &   \verb|T|  &  \verb|T|   &    \verb|T|    & \verb|T|\\
          \verb|T| & \verb|F| & \verb|F| &   \verb|F|  &  \verb|T|   &    \verb|F|    & \verb|T|\\
          \verb|F| & \verb|T| & \verb|T| &   \verb|F|  &  \verb|T|   &    \verb|T|    & \verb|F|\\
          \verb|F| & \verb|F| & \verb|T| &   \verb|F|  &  \verb|F|   &    \verb|T|    & \verb|T|\\
    \end{tabular}
\end{center}
你以前可能在其他数学课程中见过真值表,但即使没有见过,也不必担心!其主要思想如下:每一列对应一个特定数学陈述及其相应的真值。每行对应\emph{分配}给成分陈述 $P$ 和 $Q$ 的特定真值。

请注意,上面真值表有 $4$ 行,因为 $P$ 和 $Q$ 可以分别具有两个真值中任意一个,因此这些选择有 $4$ 种可能的组合。读取特定行时,我们根据前两列中 $P$ 和 $Q$ 为 \verb|T| 或为 \verb|F| 得到其他陈述的相应真值。

请注意,$P \land Q$ 和 $P \lor Q$ 列遵循上面给出的定义。$P \land Q$ 列只有一个 \verb|T|,它对应于 $P$ 和 $Q$ \emph{都}为\verb|真|的情况。所有其他情况都令 $P \land Q$ 为\verb|假|。同样,$P \land Q$ 列只有一个 \verb|F|,它对应于 $P$ 和 $Q$ \emph{都}为\verb|假|的情况。所有其他情况都令 $P \lor Q$ 为\verb|真|。

为什么最后两列的真值是这样的?假设我声称``如果你努力学习,那么你将在这门课程中获得 A''。这里,$P$ 是``你努力学习'',$Q$ 是``你会得到 A''。你什么时候会大骂我是\emph{骗子}?你什么时候会宣布我说的是实话?当然,如果你努力学习并获得了 A,我说的就是实话。反之,如果你努力学习却没有得到 A,那么我就是骗了你。不过,如果你没努力学习,那么无论发生什么,你都\emph{不能说我是骗子}!我的声明不包括不努力学习这种情况;我的前提是你会努力学习!因此,我没有说谎,所以根据排中律,我\emph{确实}说的是实话。非谎言即真理。

这种情况($P \implies Q$ 列的第三行和第四行为\verb|真|)被称为\textbf{错误假设}。当 ``$\implies$'' 左边的陈述不成立时,我们不在该主张的讨论范围内,因此我们不能断言该主张为\verb|假|。因此,该主张必然为\verb|真|(同样,根据排中间律)。

让我们对该符号进行恰当的定义,然后考虑更多的例子来说明这个定义。

\begin{definition}
    我们在数学陈述之间使用符号 ``$\implies$'' 来表示``如果……那么''或``蕴涵''。例如,我们将 ``$P \implies Q$'' 读作``如果 $P$,则 $Q$''或``$P$ 意味着 $Q$''。

    这称为\dotuline{条件语句}。

    假设每当 $P$ 成立时 $Q$ 也成立,``$P \implies Q$'' 的真值为\verb|真|。

    仅当 $P$ 为\verb|真| 而 $Q$ 为\verb|假| 时,真值才为\verb|假|。

    我们将 $P$ 称为条件陈述的\dotuline{假设},将 $Q$ 称为\dotuline{结论}。
\end{definition}

定义中关键词``每当''揭示了为什么\emph{错误假设}案例有意义。当我们知道 $P$ 为真并且可以推断出 $Q$ 也为真时,我们就可以声明 $P \implies Q$ 为\verb|真|。如果 $P$ 一开始就不为真,我们就不能声明 $P \implies Q$ 为假。只有当 $Q$ 不一定从 $P$ 得出时,即存在假设 $P$ 为\verb|真|但结论 $Q$ 为\verb|假|的情况时,我们才能说 $P \implies Q$ 为假。

\subsubsection*{示例}

这里有几个例子可以帮助你理解这个想法:
\begin{align*}
    (1 + 3 = 4) \implies (\forall x \in \mathbb{R} \centerdot x^2 \ge 0)  \qquad &\text{真} \\
    (1 + 3 = 5) \implies (\forall x \in \mathbb{R} \centerdot x^2 \ge 0)  \qquad &\text{真} \\
    (1 + 3 = 5) \implies (\text{亚伯拉罕·林肯还活着})  \qquad &\text{真} \\
    (1 + 1 = 2) \implies (0 = 1)  \qquad &\text{假} \\
    (0 = 0) \implies (\exists x \in \mathbb{R} \centerdot x^2 < 0)  \qquad &\text{假} \\
    (\text{毕达哥拉斯定理}) \implies (1 = 1)  \qquad &\text{真} \\
    (0 = 1) \implies (1 = 1)  \qquad &\text{真}
\end{align*}
请注意,第二个和第三个示例为\verb|真|,因为它们的假设 ``$1 + 3 = 5$'' 为\verb|假|。无论结论如何,整个条件陈述都必然为 \verb|真|。``$\forall x \in \mathbb{R} \centerdot x^2 \ge 0$''碰巧为\verb|真|或者``亚伯拉罕·林肯还活着''碰巧为 \verb|假|并不重要;错误的假设决定了陈述的真值必然为\verb|真|。

另外,请注意,倒数第二个例子为\verb|真|,但它并不能帮助我们确定毕达哥拉斯定理本身是否为\verb|真|!这就是我们在第 1 章中对该定理的错误``证明''所做的事情。回顾第 \ref{sec:section1.1.1} 节,特别是``证明 2''。我们假设毕达哥拉斯定理为\verb|真|,然后从该假设逻辑上导出一个为\verb|真|的陈述。这仅仅意味着我们得出了有效的结论,并不意味着假设同样有效!

这一思想非常重要,我们甚至可以马上向你展示另一个荒谬的例子。请注意,它的逻辑形式与其他错误证明完全相同:

\begin{spoof}
    假设 $1 = 0$。那么,根据 $=$ 的对称性,$0 = 1$ 同样成立。将这两个方程相加可知 $1 = 1$,为\verb|真|。因此,$0 = 1$。
\end{spoof}

这里的要点是:
\begin{center}
    总体而言,知道条件陈述为\verb|真|,并不能告诉我们有关成分命题真值的\emph{任何}信息。
\end{center}
上面第三和第七个陈述也清楚地说明了这一点;两个条件声明都为\verb|真|,但我们当然不能得出亚伯拉罕·林肯还活着或 $0 = 1$ 的结论。

\subsubsection*{``蕴涵''与``可以推导出''不同}

使用``蕴涵''一词来表示``如果……那么……''这样的条件陈述常常会引起一些混淆。我们相信这是由于``蕴涵''一词的某些含义带来的;具体来说,它似乎传达了某种\emph{因果关系}。例如,考虑以下陈述:
\[1 + 3 = 4 \implies 2 + 3 = 5\]
这是一个为\verb|真|的条件陈述,我们的大脑可能会意识到这一点,因为我们可以将假设,即 $1 + 3 = 4$,两边加 $1$,得出结论中的等式。从这个意义上说,假设的真实性似乎对结论的真实性有一定影响:我们可以\emph{直接}从一个推导出另一个。一般来说,情况不一定如此!

回顾上面给出的第一个例子:
\[(1 + 3 = 4) \implies (\forall x \in \mathbb{R} \centerdot x^2 \ge 0)\]
$1+ 3 = 4$ 这个事实与任何实数的平方都是非负数这一事实有什么关系?甚至有什么联系吗?我们其实并不关心!无论我们是否能找到一种方法直接从假设中推导出结论(以及这种推论是否存在),我们仍然可以将这个条件语句识别为\verb|真|。只有成分陈述的真值才重要。

诚然,当我们致力于证明条件陈述时,我们可能会尝试直接从一个陈述推导出另一个陈述。但请务必记住,这是我们证明策略的结果,而不是条件陈述定义方式的基本部分。由于这些原因,我们倾向于使用``如果……那么……''的形式而不是``蕴涵''来编写条件陈述。我们有时可能会使用它,并且我们确信你会在其他数学著作中看到它。但目前,在我们仍在学习逻辑陈述和逻辑连词时,我们将尽力避免它。

\subsubsection*{量化变量:同样很重要!}

在数学中,我们经常想要证明涉及变量的条件陈述。例如,我们可能想证明,在实数 $\mathbb{R}$ 的背景下,以下条件声明成立:
\[x > 1 \implies x^2 - 1 > 0\]
上面一行所写的这句话本身就是一个\textbf{变量命题},符号 ``$\implies$'' 的定义适用于它。

如果我们知道 $x > 1$ 并且 $x2 - 1 > 0$,那么我们可以声明该条件陈述为\verb|真|。如果我们知道 $x \le 1$,那么我们甚至不会关心 $x^2 - 1 > 0$ 是否为\verb|真|;就可以声明条件陈述为\verb|真|。这就是 ``$\implies$'' 的定义在这里的应用方式。

但请记住,如上所述的条件声明在技术上并不是数学陈述。我们是在实数的背景下做出的这一声明,所以如下写法才有意义
\[\forall x \in \mathbb{R} \centerdot (x > 1 \implies x^2 - 1 > 0)\]
这就是笔者最终想要表达的。这些逻辑连词---$\land , \lor$ 和 $\implies$---有意义并且可以应用于变量命题。在该范围之外,在你要组合的语句中的其他位置,必须对这些变量进行某种量化。只有这样,我们才能确信该语句是一个具有唯一真值的数学陈述。

\subsubsection*{用 ``$\lor$'' 重写 ``$\implies$''}

有一个有用且重要的想法值得一提。部分原因是我们稍后会使用它,部分原因是它可以帮助你理解条件陈述并学习如何使用它。

这个想法取决于错误假设的概念。考虑条件陈述,$P \implies Q$。如果 $P$ 不成立,则整个陈述为\verb|真|,无论 $Q$ 的真值如何。然而,如果 $P$ 成立,那么我们肯定需要 $Q$ 也成立,才能说整个陈述为\verb|真|。

这些观察使我们能够得到条件陈述得以成立的两种方式,并将这两种方式写在``或''陈述中。要么 $\neg P$ 成立(即错误假设),要么 $Q$ 成立。在任何一种情况下,条件陈述 $P \implies Q$ 都必然成立!让我们将这一观点写下来:

\begin{center}
    条件陈述 ``$P \implies Q$'' 和陈述 ``$\neg P \lor Q$'' 具有相同的真值。
\end{center}

这是\textbf{逻辑等价}的一个很好的例子,我们将在下一节中讨论这一主题。现在,我们将给出上述两种说法的真值表。请注意,无论成分陈述 $P$ 和 $Q$ 的真值如何,两种说法都具有相同的真值。除了我们上面提供的描述之外,还可以进一步验证这些陈述是等价的。

\begin{center}
    \begin{tabular}{c|c|c|c|c}
          $P$      & $Q$      & $\neg P$ & $\neg P \lor Q$ & $P \implies Q$ \\
          \hline
          \verb|T| & \verb|T| & \verb|F| &     \verb|T|    &  \verb|T|   \\
          \verb|T| & \verb|F| & \verb|F| &     \verb|F|    &  \verb|F|   \\
          \verb|F| & \verb|T| & \verb|T| &     \verb|T|    &  \verb|T|   \\
          \verb|F| & \verb|F| & \verb|T| &     \verb|T|    &  \verb|T|   \\
    \end{tabular}
\end{center}

\subsubsection*{更多示例}

让我们看更多条件声明的例子,并判断它们是对还是错。这样做有助于你更好地理解 $\implies$ 的工作原理。

然后,我们将转向证明\emph{策略},并讨论如何使用逻辑连词和量词正式且严格地证明此类主张。

\begin{example}
    我们将从一个``现实世界''中的例子开始,以习惯所涉及的逻辑。在这个例子中,假设我们所在的班级只在周一、周三和周五安排正式讲座。你会注意到,我们将采用两个陈述 $P$ 和 $Q$,并考虑这些陈述及其否定形式的所有四种可能组合,以构成条件陈述。
    \begin{itemize}
        \item ``如果今天有讲座,那么今天就是工作日。''\\
        (注:这句话中有一些\emph{隐性量化}。我们实际上是在说``对于周历中的所有自然日 $d$,如果我们在 $d$ 天有讲座,那么 $d$ 就是工作日。''我们认为上面这句话更能简洁地表达主要思想,所以将使用了更简明的版本。请记住,这是该句子的含义,在下面的讨论中,我们将考虑该量化的不同情况。)\\
        这可以通过定义 $P$ 为``今天有讲座'',$Q$ 为``今天是工作日''来将该该声明逻辑地写做 $P \implies Q$。\\
        这个声明为\verb|真|吗?请注意,陈述 $P$ 和 $Q$ 并未指定具体日期,因此,如果我们断言此声明为\verb|真|,则该事实应该独立于当前日期。也就是说,无论今天是哪一天,我们都必须证明 $P \implies Q$ 成立。让我们考虑一些情况来做到这一点:
        \begin{itemize}[label=--]
            \item 假设今天是星期六或星期天。由于这些天没有讲课,所以这个条件陈述为\verb|真|。
            \item 假设今天是星期一、星期三或星期五。今天确实有讲座,那么今天肯定是工作日,所以这个说法为\verb|真|。
            \item 假设今天是星期二或星期四。今天通常没有讲座,但即使在特殊情况下(由于某些重新安排的原因)今天有讲座,今天仍然是工作日,所以该说法依然为\verb|真|。
        \end{itemize}
        在任何可能的情况下,该声明均成立。因此,$P \implies Q$ 为真。\\
        你可能会反驳说:``为什么要费心去处理所有这些情况呢?难道我们不能说,不管今天是哪一天,假设我们有讲座,那么我们就可以断定今天一定是工作日吗?''嗯,是的,我们的确可以!你可能会说,这实际上是一个更好的策略,一条更\emph{直接}的路线。\\
        这暗示了我们未来如何证明条件声明。事实上,由于我们并不关心没有讲座的情况(\emph{错误假设}),因此我们只需要\emph{假设}我们在 $X$ 天有讲座,并\emph{推断}出 $X$ 是工作日即可。这是我们用与\textbf{直接证明}条件声明的方法。
        \item ``如果今天是工作日,那么今天有讲座。''\\
            这在逻辑上可以写做 $Q \implies P$,使用与上例相同的 $P$ 和 $Q$ 定义。\\
            这个声明为\verb|真|吗?答案当然是否定的!学期的第一个星期二没有讲座,但那天是工作日。因此,原声明在这种情况下为假!因为是星期二,所以 $Q$ 为真,但 $P$ 为假。因此,$Q \implies P$ 为\verb|假|。
        \item ``如果今天不是工作日,那么今天就没有讲座。''
            这在逻辑上可以写做 $\neg Q \implies \neg P$,使用与上例相同的 $P$ 和 $Q$ 定义。\\
            这个声明为\verb|真|吗?是的!我们可以直接证明。假设今天不是工作日;也就是说,今天是星期六或星期天。显然,大学不会变态到在周末安排讲座,所以我们有理由声明周末没有讲座,即 $\neg P$ 成立。这表明 $\neg Q \implies \neg P$ 是一个\verb|真|命题。\\
            (问题:为什么我们不需要考虑今天是工作日的情况?)
        \item ``如果今天没有讲座,那么今天就不是工作日。''
            这在逻辑上可以写做 $\neg P \implies \neg Q$,使用与上例相同的 $P$ 和 $Q$ 定义。\\
            这个声明为\verb|真|吗?让我们思考一下。如果我们假设今天没有讲座会怎样?我们能得到什么结论呢?这一定不是工作日吗?我不这么认为!也许今天是星期二,我们只是没有安排讲座而已。这表明该说法是错误的;我们有一个反例,假设 $(\neg P)$ 成立,但结论 $(\neg Q)$ 不成立。\\
            请注意,在某些情况下,$P$ 成立,$Q$ 也成立。例如,如果今天是星期六,那么我们当然没有讲座,而这又不是工作日。不过,这个\emph{具体实例}并不意味着该主张是正确的!我们需要验证\emph{所有实例}的真实性。
    \end{itemize}
\end{example}

\begin{example}
    让我们用一个更``数学''的例子来做同样的分析。在整个示例中,令 $A$ 和 $B$ 为任意集合。另外,令 $P$ 为 ``$A \subseteq B$'',令 $Q$ 为 ``$A - B = \varnothing$”。\\
    我们将像前面示例中所做的那样,考虑 $P$ 和 $Q$ 及其否定形式组合出的所有四种可能的条件陈述。
    \begin{itemize}
        \item $P \implies Q$ 为\verb|真|吗?\\
            答案是肯定的!让我们假设 $A$ 和 $B$ 满足关系 $A \subseteq B$。这意味着 $A$ 的每个元素也是 $B$ 的元素。因此,不存在 $A$ 的元素不属于 $B$ 的情况。由于 $A - B$ 是属于 $A$ 而不属于 $B$ 的元素集合,因此我们得出结论:不存在这样的元素,因此 $A - B = \varnothing$。
        \item $Q \implies P$ 为\verb|真|吗?\\
            答案是肯定的!假设 $A - B = \varnothing$。这意味着 $A$ 中没有元素不是 $B$ 的元素。(仔细思考一下。)换句话说,任意元素 $x \in A$ 都不具有 $x \notin B$ 的属性(或者 $x \in A - B$ 而 $A - B = \varnothing$); 因此,必然有 $x \in B$。而这正是 $A \subseteq B$ 的定义!每当 $x \in A$ 时,我们也得出 $x \in B$ 的结论。这就证明了 $Q \implies P$ 成立。
        \item $\neg Q \implies \neg P$ 为\verb|真|吗?\\
            这个比较难搞清楚。让我们假设 $\neg Q$ 成立;这意味着 $A - B \ne \varnothing$。也就是说,存在某个元素 $x$ 满足 $x \in A$ 且 $x \notin B$。那么自然 $A \nsubseteq B$,因为我们已经确定了 $A$ 中的一个元素不是 $B$ 的元素(而 $\subseteq$ 关系表明:$A$ 的每个元素都是 $B$ 的元素)。因此,$\neg Q \implies \neg P$ 为\verb|真|。
        \item $\neg P \implies \neg Q$ 为\verb|真|吗?\\
            同理,让我们写下 $\neg P$ 的含义。说 $A \nsubseteq B$ 意味着存在某些元素 $x \in A$ 同时满足 $x \notin B$。(这也是我们在前面的情况中使用的。)好吧,这告诉我们什么?考虑集合 $A - B$。它有元素吗?是的,它至少有元素 $x$!由于 $x \in A \land x \notin B$,我们可以说 $x \in A - B$。因此,$A - B \ne \varnothing$,所以我们得出 $\neg P \implies \neg Q$ 为\verb|真|。
    \end{itemize}
\end{example}

\subsubsection*{关于 ``$\implies$'' 的观察和事实}

上面我们看了一些判断条件陈述真值的练习。从我们讨论的例子中你应该注意到,知道 $P \implies Q$ 成立并\textbf{不能}告诉我们任何关于 $Q \implies P$ 的信息。在上面的两个例子中,$P\implies Q$ 为\verb|真|;然而,$Q \implies P$ 在一个示例中为 \verb|真|,而在另一个示例中为\verb|假|。即使我们知道 $P \implies Q$ 的真值,我们也无法肯定地得出 $Q \implies P$ 的真值。这个想法非常重要,我们将在下一小节中讨论它。

现在,让我们对 ``$\implies$'' 连词再做一些评论。

\begin{itemize}
    \item 请记住,给定数学陈述 $P$ 和 $Q$,语句 ``$P \implies Q$'' 本身就是另一个数学陈述。它具有真值。该真值取决于 $P$ 和 $Q$(按照我们上面定义的方式),但它没有告诉我们有关 $P$ 和 $Q$ 的真值的任何信息。因此,如果你只写下如下声明
    \[\text{Blah blah} \implies \text{Yada yada}\]
    我们不知道你是否想断言``Blah blah''或``Yada yada''是真是假!对于数学家而言,这只是在说:
    \begin{center}
        条件陈述```Blah blah'蕴涵`Yada yada'''为\verb|真|。
    \end{center}
    如果你想做出某种推论或演绎,需要使用某些辅助性的单词和句子来表明这一点。比如:
    \begin{center}
        $P \implies Q$ 因为……

        同时 $P$ 成立,因为……

        因此 $Q$ 成立。
    \end{center}
    如果你以前学过形式逻辑,或者在哲学课上见过这种类型的论证,那么你知道这叫\textbf{分离规则}。
    \item \textbf{错误假设}带来为\verb|真|的条件陈述这一想法有点怪异。我们意识到了这一点。这是排中率的直接后果。在错误假设下,我们不能说整个陈述为\verb|假|,所以它一定为\verb|真|,因为真值必须是其中之一。
    \item 请记住,我们始终可以通过将其转换为``或''陈述来编写不带 ``$\implies$'' 符号的条件陈述。
        \begin{center}
            陈述 ``$P \implies Q$'' 与 ``$\neg P \lor Q$'' 始终具有相同的真值。
        \end{center}
\end{itemize}

\subsubsection*{逆命题和逆否命题}

让我们为与给定条件陈述相关的不同类型的条件陈述起一些名称。后面我们会经常用到它们。

\begin{definition}
    令 $P$ 和 $Q$ 为数学陈述。考虑``原''命题 $P \implies Q$。

    我们将 $Q \implies P$ 称为原始命题的\dotuline{逆命题}。

    我们将 $\neg Q \implies \neg P$ 称为原始命题的\dotuline{逆否命题}。
\end{definition}

通过我们在上一小节中的观察,我们知道\textbf{逆命题}\emph{不一定}具有与原命题相同的真值。我们将在下一节中看到(并证明)的是,\textbf{逆否命题}总是与原命题具有\emph{相同的}真值。(这就是\textbf{逻辑等价}的概念,我们将在下一节中详细讨论。)

你可能想知道为什么我们需要这些术语。原因在于,由于可以证明逆否命题与原命题\emph{逻辑等价},因此当我们证明条件陈述时,这就产生了一种有效的证明方法。我们很快就会学习到它。这就是我们使用逆否命题的原因。

逆命题很有趣,因为它的真值不一定与原命题的真值相关:即使知道原命题为\verb|真|,逆命题可能为\verb|真|,也可能为\verb|假|。因此,每当我们证明命题 $P \implies Q$ 为\verb|真|时,数学家(可能)会立即想知道,``反过来也成立吗?'' 这是一个很自然的问题,每当你面对条件陈述时这个问题都值得思考。(事实上,如果你在一个数学家聚会上,听到有人谈论``如果……那么……''这样的陈述,你都应该问一句:``反过来也成立吗?''这样会给大家留下深刻印象。)

逆命题也是日常生活中常见的一类逻辑谬误。也许你正在与朋友的论证 $A \implies B$。结果他们反驳道:``好吧,$B$ 并不一定意味着 $A$!你的说法是错误的!'' 你是否曾因这种情况而感到沮丧?你可能会忍不住大喊:``那又怎样?我并不是想说 $B \implies A$ 是否成i。我要谈论的是 $A \implies B$。你……''(我们会在变得刻薄之前打断自己的话。)无论你的朋友是否正确,知道逆命题的真值并不能告诉你任何有关原命题真值的信息。你应该让他们明白这一点!下次遇到这种情况,你只需说:``你说的是逆命题的情况,这与我的主张在逻辑上没有必然联系。''

现在我们已经定义了所有必需的逻辑符号并看过了一些示例,是时候更进一步在证明中使用它们了!但首先,简要介绍一下集合运算,然后是一些练习问题。

\subsection{回顾:集合运算与逻辑连词}\label{sec:section4.5.4}

回顾一下 \ref{sec:section3.4} 和 \ref{sec:section3.5} 节,我们定义了子集和集合运算。所有这些定义都使用了一些逻辑思想,但当时我们是用自然语言书写的,依靠的是我们的集合直觉和逻辑知识。我们现在可以使用量词和连词重写它们!

首先,回顾一下\textbf{子集}的定义。如果以下条件成立,我们就写做 $A \subseteq B$:每当 $x \in A$ 时,我们也可以说 $x \in B$。注意关键词``每当'',它既表示\emph{全称量化}又表示\emph{条件陈述}。想想如何使用这些概念重写 $A \subseteq B$ 的定义,然后继续阅读我们的版本……

\begin{definition}
    设 $A, B, U$ 为集合,其中 $A, B \subseteq U$(即 $U$ 为全集)。我们说 $A$ 是 $B$ 的\dotuline{子集},并写作 $A \subseteq B$,当且仅当
    \[\forall x \in U \centerdot x \in A \implies x \in B \]
\end{definition}

这是合理的,因为它断言了我们在上一段中写的``每当''陈述:每当 $x \in A$ 时,我们也必然能够得出 $x \in B$ 的结论;``如果 $x \in A$,则 $x \in B$'' 必然成立。

再回顾一下我们给出的集合运算的定义。尝试使用逻辑符号为这些定义编写你自己版本的定义,然后在再阅读我们的版本。想想它们为什么合理,如何表达相同的基本想法。

\begin{definition}
    设 $A, B, U$ 为集合,其中 $A, B \subseteq U$(即 $U$ 为全集)。则
    \begin{align*}
        A \cap B &= \{x \in U \mid x \in A \land x \in B\} \\
        A \cup B &= \{x \in U \mid x \in A \lor x \in B\} \\
        A - B &= \{x \in U \mid x \in A \land \neg (x \in B)\} = \{x\in U \mid x \in A \land x \notin B\} \\
        \overline{A} &= \{x \in U \mid \neg (x \in A)\} = \{x \in U \mid x \notin A\}
    \end{align*}
\end{definition}

我们还可以重新定义集合的划分。这将用到逻辑连词,也会涉及索引集以及如何用量词定义它们。我们所学到的一切都在这里汇集!

\begin{definition}
    设 $A$ 为集合。 $A$ 的\dotuline{划分}是两两不相交且并集为 $A$ 的集合的集合。

    也就是说,分区由满足以下条件的索引集 $I$ 和非空集 $S_i$(定义在每一个 $i \in I$ 上)构成:
    \begin{enumerate}[label=(\arabic*)]
        \item $\forall i \in I \centerdot S_i \subseteq A$
        \item $\forall i, j \in I \centerdot i \ne j \implies S_i \cap S_j = \varnothing$
        \item $\displaystyle{\bigcup_{i \in I} S_i = A}$
    \end{enumerate}
\end{definition}

回顾一下定义 \ref{def:definition3.6.9},看看我们最初是如何定义划分的。你看到我们在这里如何只使用逻辑符号来表达同样的事情了吗?

\clearpage

\subsection{问题与练习}

口头或书面简要回答以下问题。这些题目全都基于你刚刚阅读的部分,因此如果你无法想起特定的定义、概念或示例,请返回重新阅读相应部分。确保自己在继续之前可以自信地回答这些问题,这将有助于你的理解和记忆!

\begin{enumerate}[label=(\arabic*)]
    \item $\land$ 和 $\lor$ 有什么区别?
    \item $\land$ 和 $\cap$ 有什么区别?\\
        $\lor$ 和 $\cup$ 有什么区别?
    \item 写出陈述 $P \implies Q$ 的真值表。
    \item 为什么 $P \implies Q$ 和 $\neg P \lor Q$ 是逻辑等价的?
    \item 条件陈述的逆命题是什么?\\
        条件陈述的逆否命题是什么?
    \item 条件陈述的真值与其逆命题是否相关?
\end{enumerate}

\subsubsection*{试一试}

尝试回答以下简答题。这些题目要求你实际动笔写一写,或(对朋友/同学)口头描述一些东西。目的是让你练习使用新概念、定义和符号。别担心,这些题本来就很简单。确保能够解决这些问题将对你有所帮助!

\begin{enumerate}[label=(\arabic*)]
    \item 对于以下每个句子,使用逻辑符号重写并确定其为\verb|真|还是\verb|假|。
        \begin{enumerate}[label=(\alph*)]
            \item 每个整数要么严格为正,要么严格为负。
            \item 对于任意给定的实数,都存在一个严格大于它的自然数。
            \item 对于每个实数,如果它是负数,那么它的立方也是负数。
            \item $\mathbb{Z}$ 的一个子集具有以下性质:每当一个数字是该集合的元素时,它的平方也是该集合的元素。
            \item 存在一个既是偶数又是奇数的自然数。
        \end{enumerate}
    \item 将以下每个 $\forall$ 命题重写为条件陈述,并确定它为\verb|真|还是\verb|假|。
        \begin{enumerate}[label=(\alph*)]
            \item $\forall x \in \{y \in \mathbb{N} \mid \exists k \in \mathbb{Z} \centerdot y = 2k\} \centerdot x^2 \in \{y \in \mathbb{N} \mid \exists k \in \mathbb{Z} \centerdot y = 2k\}$
            \item $\forall x \in \{y \in \mathbb{N} \mid \exists k \in \mathbb{Z} \centerdot y = 2k + 1\} \centerdot x^2 \in \{y \in \mathbb{N} \mid \exists k \in \mathbb{Z} \centerdot y = 2k + 1\}$
            \item $\forall t \in \{z \in \mathbb{R} \mid z^2 > 4\} \centerdot t > 2$
        \end{enumerate}
    \item 使用集合构建符将以下每个条件陈述重写为 $\forall$ 声明,并确定其为\verb|真|还是\verb|假|。
        \begin{enumerate}[label=(\alph*)]
            \item $\forall x \in \mathbb{R} \centerdot x > 3 \implies x^2 < 9$
            \item $\forall x \in \mathbb{R} \centerdot x < 3 \implies x^2 < 9$
            \item $\forall t \in \mathbb{R} \centerdot t^2 - 6t + 9 \ge 0 \implies t \ge 3$
        \end{enumerate}
    \item 们定义以下变量命题:
        \begin{align*}
            P(x) &= \frac{1}{2} < x \\
            Q(x) &= x < \frac{3}{2} \\
            R(x) &= x^2 = 4 \\
            S(x) &= x + 1 \in \mathbb{N} 
        \end{align*}
        对于以下每个陈述,确定它为\verb|真|还是\verb|假|。
        \begin{enumerate}[label=(\alph*)]
            \item $\forall x \in \mathbb{N} \centerdot P(x)$
            \item $\forall x \in \mathbb{N} \centerdot Q(x) \implies P(x)$
            \item $\forall x \in \mathbb{Z} \centerdot Q(x) \implies P(x)$
            \item $\exists x \in \mathbb{N} \centerdot \neg S(x) \lor R(x)$
            \item $\exists x \in \mathbb{Z} \centerdot R(x) \land \neg S(x)$
            \item $\forall x \in \mathbb{R} \centerdot R(x) \implies S(x)$
            \item $\exists x \in \mathbb{R} \centerdot P(x) \land S(x)$
            \item $\forall x \in \mathbb{Z} \centerdot R(x) \implies \big(P(x) \lor Q(x)\big)$
        \end{enumerate}
    \item 对于以下每个条件陈述,用逻辑符号重写它,并写出它的逆命题和逆否命题;然后,确定所有三个命题的真值:原命题、逆命题和逆否命题。
        \begin{enumerate}[label=(\alph*)]
            \item 如果一个实数严格介于 $0$ 和 $1$ 之间,那么它的平方也是如此。
            \item 如果一个自然数是偶数,那么它的立方也是偶数。
            \item 每当一个整数是 $10$ 的倍数时,它也是 $5$ 的倍数。
        \end{enumerate}
\end{enumerate}