% !TeX root = ../../book.tex
\section{展望}

在本章中,我们向你介绍了\textbf{数学归纳法}的概念。我们看了一些例题,其中归纳过程指导了我们的解题思路,然后我们描述了如何通过\textit{归纳证明}来\textit{严格验证}该解题思路。凭借迄今为止我们掌握的数学技术和概念,我们必须依靠非技术性的类比来向你描述这个过程。某种程度上,这就像让朋友向你描述如何挥高尔夫球杆,即使你以前从未打过高尔夫球。当然,它们可以为你提供一些关于挥杆“感受就像是什么”的心理想象,但如果不亲自练习,你如何真正理解高尔夫挥杆的机制呢?如何学习怎样调整你的挥杆动作,或辨别一号木杆、五号铁杆和沙坑挖起杆之间的区别呢?通过研究底层机制并刻意练习,我们希望更好地理解数学归纳法,以便将来我们可以适当地使用它,识别它适用的情况,并最终学习如何使其\textit{适应}其他情况。当然,记住多米诺骨牌的类比会有助于引导我们的直觉,但我们也应该记住,这不是严格的数学。它也不能完美地描述我们讨论的其中几个例子,在这些例子中,倒下的多米诺骨牌不仅取决于紧随其后的多米诺骨牌,还取决于它之前的其他几张多米诺骨牌。

在下一章中,我们将探讨严格陈述和证明数学归纳法作为证明技术所需的一些相关概念。具体来说,我们将研究数理逻辑的一些思想,研究如何将复杂的数学陈述和定理分解为更小的组成部分,以及如何从基本构建模块中构建起有趣且复杂的陈述。在此过程中,我们将引入一些新的符号和速记符,使我们能够将一些冗长的陈述压缩成简洁(且精确)的数学语言。有了这些之后,我们将探索一些更基本的证明策略,然后将其应用于本课程中所做的所有其他事情,包括归纳技术本身!我们还将研究集合论的一些思想,集合论是数学的一个分支,构成了所有其他分支的基础。这对于将来组织我们的想法非常有用,而它也将帮助我们以严格的方式定义自然数。有了这两个数学分支的一些概念和知识,我们就可以在坚实的基础上建立数学归纳法,并继续正确地使用它。