% !TeX root = ../../book.tex
\section{定义归纳}

为了将数学归纳法定义为证明技术,我们想强调,上一节中的例子使用了问题结构的某些直观概念来给出问题的``解'',我们在\textit{解}上加了引号,表明我们还没有正式证明它。从这个意义上讲,我们提出以下问题:如果\textit{给定}我们前面推导的公式并要求验证它怎么办?如果我们没有通过任何直观的步骤来推导公式,只是有人告诉我们它是正确的怎么办?我们如何验证他们的说法?之所以问这个问题,是因为我们现在确实面临着这种情况,除非告诉我们公式的人采用与我们相同的直觉论证。

假如一个持怀疑态度的朋友说:``嘿,我听说过一个计算前 $n$ 个自然数平方和的公式。有人告诉我,它们加起来等于 $\frac{1}{6}n(n+1)(2n+1)$。我验证了前两个自然数,全都正确,所以它一定是正确的。应该传播出去!'' 作为一个理性思考者,同时也是好朋友的身份,你点点头说:``我确实听说了,但让我们确保这个公式对每个数字都是正确的。'' 你将如何进行?你的朋友说的没错,前几个值确实``完美匹配'':

\begin{align*}
    1^2 &= \enspace 1 = \frac{1}{6}(1)(2)(3) \\
    1^2 + 2^2 &= \enspace 5 = \frac{1}{6}(2)(3)(5) \\
    1^2 + 2^2 + 3^2 &= 14 = \frac{1}{6}(3)(4)(7) \\
    1^2 + 2^2 + 3^2 + 4^2 &= 30 = \frac{1}{6}(4)(5)(9)
\end{align*}
如果我们愿意的话,我们甚至可以手动检验 $n$ 的更大的值:
\[1^2 + 2^2 + 3^2 + 4^2 + 5^2 + 6^2 + 7^2 + 8^2 + 9^2 + 10^2 = 385 = \frac{1}{6}(10)(11)(21)\]
但请记住,该公式声称对于\textit{任意} $n$ 值都有效。手动检验每个结果将花费大量时间,因为自然数有\textit{无穷}多个。无论我们检验多少个独立的 $n$ 值,总会有更大的值,我们怎么\textit{知道}公式对于某些大值不会失效?从数学和时间上来讲,我们需要一个更加\textit{有效}的方法,以某种方式只需几步即可验证所有 $n$ 值。我们先在心里埋下一颗种子(这是即将推出的数学归纳法的严格版本),在这里我们将从广义上解释该过程是如何工作的。

\subsection{类比多米诺骨牌}

假设我们有一副多米诺骨牌。这是一副特殊的多米诺骨牌,里面有无穷多的骨牌!我们可以想象在上面有任何我们想要的内容,而不是标准的点数。我们还假设它们沿着无限延申的桌面排列成无限长的一行。从侧面看多米诺骨牌,可以看到每张牌下面有一个标签,以便知道其在行中的位置:

\begin{center}
    \begin{tikzpicture}
        \foreach \x in {1,...,5}
        {
            \pic [fill=white] at (\x, 0, 0) {annotated cuboid={width=3, height=30, depth=10}};
            \node[below] at (\x, -3){\tiny $n=\x$};
        }
        \node[anchor=center] at (6.5, -1.5){\LARGE $\dots \cdot$};
        % \pic [very thick,densely dashed,draw=blue] at (5,0) {annotated cuboid={width=30, height=5, depth=10, opacity=0.2}};
    \end{tikzpicture}
\end{center}

对于这个特定的例子,要验证公式
\[\sum_{k=1}^{n}k^2 = \frac{1}{6}n(n+1)(2n+1)\]
我们想象每张多米诺骨牌上都写有一个特定的``事实''。具体来说,我们可以想象第一张多米诺骨牌上写有表达式
\[\sum_{k=1}^{1}k^2 = \frac{1}{6}(1)(1)(3)\]
第二张多米诺骨牌上写有表达式
\[\sum_{k=1}^{2}k^2 = \frac{1}{6}(2)(3)(5)\]
总的来说,我们想象第 $n$ 张多米诺骨牌上写着以下``事实'':
\[\sum_{k=1}^{n}k^2 = \frac{1}{6}n(n+1)(2n+1)\]
由于多米诺骨牌本来就是要互相推倒、互相撞倒的,所以让我们假设每当多米诺骨牌倒下时,就意味着上面的相应``事实''是一个\textit{真实陈述}。这就将多米诺骨牌的物理解释与我们推导公式有效性的数学解释联系起来。

我们手动检验了 $n = 1$ 的求和:$1^2=\frac{1}{6}(1)(2)(3)$,因此,第一张多米诺骨牌上的事实是一个真实陈述,所以我们知道第一张多米诺骨牌一定会倒下。我们还手动检验了 $n = 2$ 的求和,因此我们知道第二张多米诺骨牌也会倒下:

\begin{center}
    \begin{tikzpicture}
        \foreach \x in {1,2}
        {
            \pic [fill=white, rotate=-30, anchor=south] at (\x, -0.15, 0) {annotated cuboid={width=3, height=32, depth=10}};
            \node[below] at (\x-1.5, -3){\tiny $n=\x$};
        }
        \foreach \x in {3,4,5}
        {
            \pic [fill=white] at (\x, 0, 0) {annotated cuboid={width=3, height=30, depth=10}};
            \node[below] at (\x, -3){\tiny $n=\x$};
        }
        \node[anchor=center] at (6.5, -1.5){\LARGE $\dots \cdot$};
        % \pic [very thick,densely dashed,draw=blue] at (5,0) {annotated cuboid={width=30, height=5, depth=10, opacity=0.2}};
    \end{tikzpicture}
\end{center}
然而,继续这样下去会让我们回到与之前相同的问题:我们不想检查\textit{每一张}多米诺骨牌以确保它倒下。我们真的很想封装多米诺骨牌的物理概念 --- 即,当多米诺骨牌倒下时,会碰倒下一张多米诺骨牌,依此类推 --- 并以某种方式将相邻多米诺骨牌上的``事实''联系起来。

让我们看看前两张多米诺骨牌的情况。知道骨牌 $1$ 倒下,我们能在不重写所有求和项的情况下保证骨牌 $2$ 倒下吗?两块多米诺骨牌上的陈述有何关联?每个陈述都是自然数的平方和,第二张多米诺骨牌上的陈述正好多了一项。因此,既然已经知道骨牌 $1$ 已经倒下,我们可以使用骨牌 $1$ 上写的\textit{真实陈述}来\textit{验证}骨牌 $2$ 上陈述的真实性:
\[\sum_{k=1}^{2}k^2 = 1^2+2^2=1+2^2=5=\frac{1}{6}(2)(3)(5)\]
现在,这可能看起来有点愚蠢,因为我们节省的唯一``工作''是不必``进行算术运算''来计算 $1^2 = 1$。让我们在数字较大的情况下使用此过程,以便充分说明这种做法的好处。\textit{假设}骨牌 $10$ 已经倒下。(如果你担心这个假设,我们在前面给出了完整的求和计算,你可以在那里验证。)这意味着我们\textit{知道}
\[\sum_{k=1}^{10}k^2 =\frac{1}{6}(10)(11)(21)=285\]
是一个\textit{真实陈述}。我们用它来验证骨牌 11 上写的陈述,即
\[\sum_{k=1}^{11}k^2 =\frac{1}{6}(11)(12)(23)\]
骨牌 $11$ 上的求和公式有 $11$ 项,前 $10$ 项正是骨牌 $10$ 上的求和!由于我们对该求和有所了解,因此我们只需将第 $11$ 项与其他求和项分开,并应用我们对其他项的已知信息:

\begin{align*}
    \sum_{k=1}^{11}k^2 &= (1^2+2^2+\dots+10^2)+11^2\\
    &=\sum_{k=1}^{10}k^2+11^2\\
    &=385+121\\
    &=506\\
    &=\frac{1}{6}3036=\frac{1}{6}(11)(12)(23)
\end{align*}
看看我们节省的工作!如果我们已经对求和的前 $10$ 项有所了解,为什么还要费力去计算它们呢?

现在,想象一下我们是否可以\textit{同时}对\textit{所有} $n$ 值执行此过程!也就是说,假设我们可以证明,只要多米诺骨牌 $n$ 倒下,我们就可以\textit{保证}多米诺骨牌 $(n + 1)$ 倒下。这会告诉我们什么?再思考一下无穷多的多米诺骨牌。我们知道骨牌 1 会倒下,因为我们手动检验了该值。然后,因为我们对所有 $n$ 值都验证了``骨牌 $n$ 撞倒骨牌 $(n + 1)$''的步骤,所以我们知道 骨牌 $1$ 会撞倒骨牌 $2$,骨牌 $2$ 又撞倒骨牌 $3$,骨牌 $3$ 有撞倒骨牌 $4 \dots$,一直持续下去,整行多米诺骨牌都会倒下!本质上,我们可以将整行多米诺骨牌分解为\textit{两}步:

\begin{enumerate}[label=(\arabic*)]
    \item 确保第一张多米诺骨牌倒下;
    \item 确保每张多米诺骨牌都能撞倒后面的多米诺骨牌。
\end{enumerate}
只需这两个步骤,我们就可以\textit{保证}每一张多米诺骨牌都会倒下,从而\textit{证明}上面写的所有事实都是真的。这将证明我们推导的公式对于\textit{每一个}自然数 $n$ 都有效。

我们已经完成了步骤(a),所以现在我们必须完成步骤(b)。我们已经在前面的段落中针对特定案例(骨牌 $1$ 推倒骨牌 $2$骨牌 $10$ 推倒骨牌 $11$)执行了此操作,因此让我们试着遵循这些案例的步骤将其推广到任意 $n$ 值。我们\textit{假设},对于某个\textit{特定}但\textit{任意}的 $n$ 值,多米诺骨牌 $n$ 会倒下,这告诉我们方程
\[\sum_{k=1}^{n}k^2=\frac{1}{6}n(n+1)(2n+1)\]
为\textit{真实陈述}。现在,我们想要将其与骨牌 $(n + 1)$ 上的陈述联系起来,并应用上面等式中给出的信息。让我们像之前一样,将 $n+1$ 项的和写为 $n$ 项之和加上最后一项:
\[\sum_{k=1}^{n+1}k^2 = (1^2+2^2+\dots+n^2+(n+1)^2)=\sum_{k=1}^{n}k^2+(n+1)^2\]
接下来,我们可以利用多米诺骨牌 $n$ 已经倒下的假设(这告诉我们骨牌上的事实为真)得到
\[\sum_{k=1}^{n+1}k^2 = \frac{1}{6}n(n+1)(2n+1)+(n+1)^2\]
这与骨牌 $(n+1)$ 上的事实一样吗?我们先来看看这个式子是什么,然后再进行比较。骨牌 $(n + 1)$ 上的``事实''与骨牌 $n$ 上的事实类似,只是将所有 ``$n$'' 的地方都替换为 ``$n + 1$'':
\[\sum_{k=1}^{n+1}k^2 = \frac{1}{6}\big(n+1\big)\big((n+1)+1\big)\big((2(n+1)+1)\big)=\frac{1}{6}(n+1)(n+1)(2n+3)\]
目前还不清楚我们推导出的表达式是否实际上等于上面的式子。我们可以尝试化简我们推导出的表达式,并将其分解为``看起来像''上面表达式的新表达式,但展开两个表达式并比较所有项可能会更容易。(这是出于这样的一般思想:展开因式分解后的多项式比进行因式分解要容易得多。)对于第一个表达式,我们得到

\begin{align*}
    \frac{1}{6}n(n + 1)(2n + 1) + (n + 1)^2 &=\frac{1}{6}n(2n^2 + 3n + 1) + (n^2 + 2n + 1)\\
    &= \frac{1}{3}n^3 + \frac{1}{2}n^2 + \frac{1}{6}n + n^2 + 2n + 1 \\
    &= \frac{1}{3}n^3 + \frac{3}{2}n^2 + \frac{13}{6}n + 1
\end{align*}
对于第二个表达式,我们得到
\begin{align*}
    \frac{1}{6}(n+1)(n + 2)(2n + 3) &=\frac{1}{6}(n+1)(2n^2 + 7n+6)\\
    &= \frac{1}{6}\big[(2n^3 + 7n^2 + 6n) + (2n^2 + 7n + 6)\big] \\
    &= \frac{1}{3}n^3 + \frac{3}{2}n^2 + \frac{13}{6}n + 1
\end{align*}
看呐,它们是相等的!此外,请注意,这比尝试整理其中一个表达式并将其``变形''为另一个表达式要容易得多。我们通过展开两式并最终找到相同的表达来证明它们是相同的。现在,让我们回顾并评估我们所取得的成果:

\begin{enumerate}
    \item 我们将证明公式
    \[\sum_{k=1}^{n+1}k^2 = \frac{1}{6}n(n+1)(2n+1)+(n+1)^2\]
    对于\textit{所有} $n$ 值的有效性类比为推倒无穷多的多米诺骨牌。
    \item 我们通过手工检验与该情况相对应的公式来验证多米诺骨牌 $1$ 会倒下。
    \item 我们通过\textit{假设}骨牌 $n$ 上的事实为真,并使用该信息来证明骨牌 $(n + 1)$ 上的事实也一定为真,从而证明了骨牌 $n$ 会倒下并撞倒骨牌 $(n+1)$。
    \item 这保证了所有多米诺骨牌都会倒下,因此该公式对于\textit{所有} $n$ 值都成立!
\end{enumerate}
这项技术有说服力吗?你认为我们已经\textit{严格证明}了该公式对于所有自然数 $n$ 都有效吗?如果有一个 $n$ 值使公式不成立怎么办?这对我们的多米诺骨牌体系意味着什么?

请记住,这里的多米诺骨牌类比只是展示归纳法如何工作的一个很好的直观指引,并不是建立在严格的数学基础上的。这将是接下来几章的目标。现在,让我们回顾一下本章中讨论的另一个示例:平面上的线。同样,在推导公式 $R(n)$ 时使用省略号很麻烦,我们希望避免使用省略号。让我们试着将多米诺骨牌类比应用于这道题。

想象一下,我们定义表达式 $R(n)$ 表示由 $n$ 条直线创建的平面中不同区域的数量,其中这些直线两两不平行,也没有三条或以上直线相交于一点。另外,想象一下,我们在骨牌 $n$ 上写下 ``$R(n) = 1 + \frac{n(n+1)}{2}$'' 这一``事实''。我们是否可以按照与上面相同的步骤来验证所有多米诺骨牌都会倒下?

首先,我们需要检验骨牌 $1$ 是否一定会倒下。这相当于验证以下陈述:``$R(1) = 1+\frac{1(2)}{2} = 1+1 = 2$'' 是否为真?这当然为真,我们之前验证过。一条线会将平面分为两个区域。其次,我们需要证明对于\textit{任意} $n$ 值,骨牌 $n$ 都会撞倒骨牌 $(n + 1)$。也就是说,我们\textit{假设} ``$R(n) = 1 + \frac{n(n+1)}{2}$'' 对于某个 $n$ 值是成立,然后\textit{证明} ``$R(n + 1) = 1 + \frac{(n+1)(n +2)}{2}''$ 也必然成立。我们应该怎么做?让我们继续沿用之前的论证方法,将 $R(n + 1)$ 与 $R(n)$ 联系起来。向\textit{任意}具有 $n$ 条直线的图形中添加一条新的直线(符合题目对于直线的要求),通过研究其几何影响,我们证明了 $R(n+ 1) = R(n) +n+ 1$。利用这些知识和我们对骨牌 $n$ 会倒下的假设,我们可以知道
\[R(n + 1) = R(n) + n + 1 = 1 +\frac{n(n+1)}{2}+ n + 1\]
这与骨牌 $(n + 1)$ 上的表达式一样吗?让我们化简这两个表达式来验证它们是否相同。可得
\[1 +\frac{n(n+1)}{2}+ n + 1=2+n+\frac{n^2+n}{2} = \frac{1}{2}n^2+\frac{3}{2}n+2\]
和
\[1 + \frac{(n+1)(n +2)}{2} = 1+\frac{n^2+3n+2}{2} =  \frac{1}{2}n^2+\frac{3}{2}n+2\]
瞧,它们是相同的!因此,我们证明了,对于\textit{任意} $n$ 值,骨牌 $n$ \textit{保证}撞倒骨牌 $(n+1)$。

想想我们用``多米诺骨牌技术''所做的事情与我们之前为推导出刚刚证明的表达式所做的事情之间的差异。我们在本节中用过省略号吗?为什么这种证明方式更好?我们曾经用过多米诺骨牌归纳技术推导过公式吗?

\subsection{其他类比}

多米诺骨牌类比非常流行,但它并不是归纳法工作方式的唯一描述。根据你的阅读内容或交谈对象,可能会学到不同的类比,或其他类型的描述。这里,我们将描述以前听说过的两个。思考这些类比本质上的等价性,这将有助于巩固你对归纳法的理解(至少就我们所开发的而言)。

\subsubsection*{神奇的数学猴子 Mojo}

想象一个无穷天梯,直矗云霄。梯子有无数级,按 $1, 2, 3$ 的顺序依次编号。我们的朋友 Mojo 恰好站在梯子旁。他是一只聪明的猴子,对数学很感兴趣,但也有点神奇,因为他真的可以爬上这个无穷天梯!

如果 Mojo 到达了阶梯上的某一级,则意味着与该数字对应的事实为真。我们怎样才能确保他爬完整个梯子?单独检查每个阶梯的效率很低。想象一下:我们必须站在地面上确保他到达第 $1$ 级,然后我们必须稍微抬起头来确保他到达了第 $2$ 级,然后是第 $3$ 级,依此类推……相反,我们在 Mojo 开始攀爬之前确认了两个细节。他要开始攀爬了吗?也就是说,他会爬上第 $1$ 级吗?如果是这样,那就太好了!另外,阶梯之间的距离是否足够近,以便无论他在哪里,\textit{总能}到达下一个阶梯?如果是这样,那就更棒了!这些与我们在多米诺骨牌类比中建立的条件完全相同。为了确保 Mojo 到达\textit{每个}阶梯,我们只需要知道他到达了第 $1$ 个阶梯,并且他总是可以到达下一个阶梯。

\subsubsection*{归纳鸭 Doug}

再来认识一下 Doug。他是一只鸭子。他喜欢面包,所以他会去每个人的院子里寻找更多的面包。这些院子都沿数学镇的归纳街而建,房子的编号是 $1, 2, 3, \dots$ 以此类推。

Doug 从 $1$ 号院子开始寻找面包。没有找到任何东西,所以他依旧很饿。还能去哪里找?隔壁还有 $2$ 号院子!Doug 朝那边走去,肚子咕咕叫。他在那里也没找到面包,所以他必须继续寻找。此时他已经知道 $1$ 号院子没有面包,所以唯一去向就是隔壁的 $3$ 号院子。我想你已经明白事情的发展方向了…… 

如果我们跟踪 Doug 的进展,我们可能想知道他最终是否到达了每一个院子。假设我们已经提前知道\textit{没人}有面包。这意味着,每当 Doug 在某个院子里时,他一定会去隔壁院子,继续寻找食物。这意味着他一定会挨家挨户地去寻找!也就是说,无论我们住在哪栋房子里,无论我们门前的数字多大,在某个时点我们一定会看到 Doug 在我们的后院闲逛。(不幸的是,他会一直饿着肚子!可怜的 Doug。)

\subsection{总结}

让我们重新思考一下前面两个示例所完成的工作以及我们给出的类比。在我们对每个题目的初步思考中,我们发现题目存在某种\textit{结构},其中一个``事实''依赖于``前一个事实''。对于立方数,我们找到了一种用 $n^3$ 的\textit{项}来表示 $(n + 1)^3$ 的方法;就平面上的线而言,我们描述了当向具有 $n$ 条直线的图形中添加一条新的直线时,会添加多少个区域。根据这些观察,我们一遍又一遍地应用这些已知的知识,直到我们得到一个我们确认的``事实'',一般是较``小''的 $n$ 值(在这两题中,$n = 1$)。这让我们能够推导出适用于\textit{任意} $n$ 值的通用公式、方程或表达式。

这项工作对于推导这些表达式来说很有趣且至关重要,但还\textit{不足以证明}这些表达式的有效性。在进行上述工作时,我们发现了归纳过程的存在,并利用其结构推导了相关表达式。这实际上有两个好处:我们找到了要证明的表达式,并且通过认识题目的归纳行为,我们意识到通过\textit{数学归纳法}来证明表达式是严谨且有效的。

对于实际的``归纳证明'',我们遵循两个主要步骤。首先,我们确定了一个``起始值'',我们可以手动检验公式/方程。其次,我们\textit{假设} $n$ 的某个特定值使得相应的公式成立,然后使用这一知识来证明相应公式对 $n + 1$ 也必然成立。在这两个步骤之间,我们可以放心地说``所有多米诺骨牌都会倒下'',因此,这些公式对于 $n$ 的所有相关值都成立。

\subsubsection*{一个问题:梯子的``尽头''是什么?}

你可能还存有疑虑,我们在这里尝试预测一下你的担忧。(我们之所以提到这一点,是因为这是一个常见的观察结果。如果你\textit{没有}考虑到这一点,请试着想象一下这个想法来自哪里。)你可能会说,``嘿,现在我想我清楚 Mojo 是如何攀登天梯了,但他如何才能真正\textit{爬到顶端}呢?这是一个无穷阶梯,对吗?那他永远无法到达顶端……不是吗?''

某种程度上,你是对的。既然这个神奇的梯子会\textit{永远}持续下去,那么它就真的没有尽头,Mojo 永远不会到达``顶端''。然而,这不是重点;我们不关心梯子的任何``\textit{顶端}''(不仅仅是因为\textit{没有}顶端)。我们只需要知道 Mojo 实际上到达了\textit{每一个可能的}阶梯。他不必超越所有人,站在梯子的顶端,俯视自己的来路。那不是目标!

我们可以这样想这个问题:假设你对我们正在证明的某些特定事实保佑浓厚的兴趣。假设这个事实是事实 $\#18,458,789,572,311,000,574,003$。(某个巨大的数字。具体是多少无关紧要。)它对应的阶梯在梯子很远很远的地方,你关心的只是 Mojo 是否能在他的旅程中到达那里。他会到达吗?……你打赌他会!这可能需要很长时间(要走多少步呢?),但在这个猴子和梯子的神奇世界里,谁会在乎时间呢!你知道他最终会到达那里,这就是你想要的。现在,想象一下,对于每个事实,在那个神奇的世界里都有一个人只关心这个事实。当然,每个人都会高兴地知道 Mojo 将在他的旅程中达到他们关心的阶梯。没有人关心他是否能登上顶端;那不是人们关心的事。与此同时,在我们这个正常的、非魔法的世界里,我们对\textit{那个}世界上的每个人最终都会高兴这一事实感到非常高兴。整个无限攀爬梯子的过程被浓缩为两步,只需要这两步,我们就可以放心,梯子上的每一个台阶都会到达。每一个编号的事实都是真的。

也可以用多米诺骨牌的类比来思考这个问题。我们是否关心多米诺骨牌是否存在某个``终点'',倒在某处墙上?当然不关心; 这条多米诺骨牌链会永远持续下去。每一张多米诺骨牌最终都会倒下,我们甚至不在乎这需要多长时间。同样地,我们知道 Doug 会到达每个院子;我们不在乎他``何时''到达\textit{某个}院子,只关心他到达了\textit{所有}院子。

\subsection{问题与练习}

\subsubsection*{提醒自己}

口头或书面简要回答以下问题。这些题目全都基于你刚刚阅读的部分,因此如果你无法想起特定的定义、概念或示例,请返回重新阅读相应部分。确保自己在继续之前可以自信地回答这些问题,这将有助于你的理解和记忆!

\begin{enumerate}[label=(\arabic*)]
    \item 多米诺骨牌、Mojo 和 Doug 类比是如何等价的?你能给出``函数''来描述它们的关系,将一种类比转换成另一种类比吗?
    \item 找一个没学过数学归纳法的朋友,试着向他描述一下数学归纳法。你发现自己使用了其中的类比吗?有帮助吗?
    \item 为什么我们对立方体的研究未能证明求和公式?为什么我们还需要完成所有这些工作?
    \item 想想多米诺骨牌的类比。多米诺骨牌永远持续下去是一个问题吗?这是否意味着有些多米诺骨牌永远不会倒下?尝试用类比来描述这意味着什么。
\end{enumerate}

\subsubsection*{试一试}

尝试回答以下简答题。这些题目要求你实际动笔写一写,或(对朋友/同学)口头描述一些东西。目的是让你练习使用新概念、定义和符号。别担心,这些题本来就很简单。确保能够解决这些问题将对你有所帮助!

\begin{enumerate}[label=(\arabic*)]
    \item 通过归纳步骤来证明该公式
    \[\sum_{k=1}^{n}k = \frac{n(n+1)}{2}\]
    \item 通过归纳步骤来证明该公式
    \[\sum_{k=1}^{n}2k-1 = n^2\]
    \item 通过归纳步骤来证明该公式
    \[\sum_{k=1}^{n}k^3 = \Bigg(\frac{n(n+1)}{2}\Bigg)^2\]
    \item 假设我们有一系列由自然数索引的事实。我们使用表达式 ``$P(n)$'' 表示第 $n$ 个事实。
    \begin{enumerate}[label=(\alph*)]
        \item 如果我们想证明对于每个自然数 $n$,\textit{每个}事实都为真,我们应该怎么做呢?
        \item 如果我们想证明只有 $n$ 为\textit{偶数}时对应的陈述才为真,那该怎么办?我们能做到吗?你能用我们给出的一个类比稍作修改来描述你的方法吗?
        \item 如果我们想证明只有 $n$ 大于等于 $4$ 时才对应的陈述才为真,那该怎么办?我们能做到吗?你能用我们给出的一个类比稍作修改来描述你的方法吗?
    \end{enumerate}
\end{enumerate}
